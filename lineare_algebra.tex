% !TeX root = matze_fruehstueckt.tex
\chapter{Vektoren, Vektorräume}

\index{Vektor|(}


\section{Wichtige Begriffe}
\begin{description}
  \item [{Vektor}]
	Gerichtete Strecke ohne Position.
  \item [{Gebundener~Vektor}] \index{Vektor!gebunden}
	Vektor mit Position, zwischen zwei Punkten: $\overrightarrow{PQ}$.
  \item [{Ortsvektor}] \index{Ortsvektor}\index{Vektor!Orts--}
	Gebundener Vektor mit Position im Ursprung.
  \item [{Parallelität}] \index{Vektor!parallel}
	$\exists\alpha : \vec{a} = \alpha\vec{b} \iff \vec{a} \| \vec{b}$
	(gleiche Richtung: $\alpha>0$, gegensätzliche Richtung: $\alpha<0$).
  \item [{Orthogonalität}] \index{Vektor!orthogonal}
	$\langle \vec{a},\vec{b} \rangle = 0
	  \quad\iff\quad
	  \vec{a}\bot\vec{b}
	  \quad\iff\quad
	  \lVert \vec{a}+\vec{b} \rVert^2 = \lVert\vec{a}\rVert^2 + \lVert\vec{b}\rVert^2$
\end{description}

\subsection{Vektorraum}

\index{Vektorraum}
Sei $K$ ein beliebiger Körper und $V$ eine nicht-leere Menge.
$V$ heißt Vektorraum, wenn für $\forall x,y\in V$ und $\forall\lambda,\mu\in K$ gilt:

\begin{itemize}
  \item Folgende Abbildungen sind definiert:
    \[ x       \oplus y : \quad V\times V \to V, \qquad(x,y)      \mapsto x\oplus y \]
    \[ \lambda \odot  x : \quad K\times V \to V, \qquad(\lambda,x)\mapsto\lambda\odot x \]
  \item $V$ ist eine kommutative Gruppe bezüglich $\oplus$.
  \item $\lambda\odot(\mu\odot x)=(\lambda\cdot\mu)\odot x$.
  \item $1\odot x=x$ (hier ist $1$ das Neutralelement bzgl.~der Multiplikation aus $K$).
  \item $\lambda\odot(x\oplus y)=(\lambda\odot x)\oplus(\lambda\odot y)$.
  \item $(\lambda+\mu)\odot x=(\lambda\odot x)\oplus(\mu\odot x)$.
\end{itemize}

Dabei nennt man die $x$, $y$ \emph{Vektoren} und die $\lambda$, $\mu$ \index{Skalar}\emph{Skalare}.
\begin{description}
  \item [Linearkombination]
        $\sum_{1 \leq i \leq n} \lambda_i \vec v_i$ heißt Linearkombination von $\vec v_1, \ldots, \vec v_n$.
  \item [linear unabhängig]
        $n$ Vektoren $\vec v_1, \ldots, \vec v_n$sind genau dann linear unabhängig, wenn $\sum_i \lambda_i \vec v_i = 0 \Rightarrow \lambda_1,\ldots,\lambda_n = 0$ ist.
  \item [Lineare Hülle]
	Die Menge aller Linearkombinationen
	\[ L=\biggl\{ \sum_i \lambda_i \vec v_i \biggm| \lambda_i \in K\biggr\} \subseteq V \]
	heißt lineare Hülle von $\vec{v}_{1},\ldots,\vec{v}_{n}$.
  \item [Erzeugendensystem]
	Gilt $L(\vec v_1,\ldots,\vec v_n)=V$, so bilden die Vektoren ein Erzeugendensystem.
  \item [Basis]
	Auch: \emph{minimales Erzeugendensystem}.
	Ist ein Erzeugendensystem, bei dem die Vektoren linear unabhängig sind.
	Daher hat eine Basis für einen Vektorraum $V$ mit $\dim(V)=n$ genau $n$ Basisvektoren.
  \item [Dimension]
	Hat ein Vektorraum $V$ eine Basis $\{ \vec v_1,\ldots,\vec v_n\}$, so ist $\dim(V)=n$.
\end{description}

\subsubsection{Untervektorraum}

\index{Unterraum}\index{Vektorraum!Unter-}\index{Untervektorraum}
Eine nicht-leere Teilmenge $U$ eines Vektorraums $V$ über einem Körper $K$ ist ein Untervektorraum, wenn gilt:
\[ \forall x,y\in U,\; \forall\lambda\in K\ :\ x+y\in U\land\lambda x\in U \]

\subsection{Skalarprodukt}
\index{Skalarprodukt}\index{Vektorraum!Skalarprodukt}
$V$ sei bel.~Vektorraum über einem Körper $K\in\{\mathbb{R},\mathbb{C}\}$ und $a,\, b,\, c\in V$ sowie $\lambda\in K$, dann heißt $\langle \ast,\ast \rangle : V\times V\to K$ Skalarprodukt falls gilt:
\begin{description}
  \item [{SP1}]
	$\langle a,b\rangle =\overline{\langle b,a\rangle}$
	[Gilt wegen fehlendem Imaginärteil auch für $K\ne\mathbb{C}$.]
  \item [{SP2}]
	$\langle a,b+c\rangle =\langle a,b\rangle +\langle a,c\rangle $
	und $\langle a+b,c\rangle =\langle a,c\rangle +\langle b,c\rangle $
  \item [{SP3}]
	$\langle \lambda a,b\rangle =\lambda\langle a,b\rangle =\langle a,\bar{\lambda}b\rangle $
	[Gilt wegen fehlendem Imaginärteil auch für $K\ne\mathbb{C}$.]
  \item [{SP4}]
	$\langle a,a \rangle = 0 \iff a=0$
	und $\langle a,a \rangle > 0 \iff a\ne0$
  \item [{Standardskalarprodukt}] $\langle \vec{a},\vec{b}\rangle :=\sum_{i} a_i b_i$
\end{description}

\subsection{\label{sub:Vektornorm}Norm}

\index{Norm}\index{Vektor!Norm}
Sei $V$ ein Vektorraum über $K \in \{\mathbb R, \mathbb C\}$ sowie $a,b \in V$ und $\lambda \in K$, dann heißt $\lVert*\rVert : V \to \mathbb R$ Norm falls $\forall a,b \in V$ und $\forall\lambda \in K$ gilt:
\begin{description}
  \item [{N1}]
	$\lVert a\rVert\geq0$
  \item [{N2}]
	$\lVert a\rVert=0 \iff a=0$
  \item [{N3}]
	$\lVert \lambda a \rVert  =  \lvert \lambda \rvert \, \lVert a \rVert$
  \item [{N4}]
	$\lVert a+b\rVert\leq\lVert a\rVert+\lVert b\rVert$ (Dreiecksungleichung)
  \item [{Standardnorm}] \index{Vektor!Betrag}\index{Vektor!Euklidische Norm}\index{Vektor!Zweiernorm}\index{Betrag}\index{Norm!Betrag}\index{Norm!Zweier--}\index{Norm!euklidische}\index{Vektor!Standardnorm}\index{Norm!Standard--}
	$\lVert a\rVert=\sqrt{\langle a,a\rangle }$ (auch \emph{euklidische Norm}, \emph{Zweiernorm} oder \emph{Betrag})
  \item [{Betragssummennorm}] \index{Vektor!Einernorm}\index{Vektor!Betragssummennorm}\index{Norm!Einer--}\index{Norm!Betragssummen--}
	$\lVert a\rVert_1=\sum_{i}\lvert a_i \rvert$ (auch \emph{Einernorm})
  \item [{Maximumnorm}] \index{Vektor!Maximumnorm}\index{Norm!Maximum--}
	$\lVert a\rVert_{\infty} = \max\bigl\{\lvert a_i \rvert \bigr\}$
  \item [{Einheitsvektor}] \index{Vektor!Einheits--}\index{Einheitsvektor}
	$\lVert a\rVert=1$
  \item [{Kanonischer~Einheitsvektor}] \index{Vektor!kanonischer Einheits--}\index{Einheitsvektor!kanonisch}
	$\lVert a\rVert=1\land\exists! a_i : a_i=1$
\end{description}
\index{Metrik}\index{metrischer Raum}\index{Raum!Metrik}\index{Isometrie}
\label{metrischer-raum}
[Die Norm stellt einen Spezialfall einer Metrik $d(x,y)=\lVert x-y \rVert$ dar; eine Abbildung $d : M \times M \to \mathbb R$ auf \emph{beliebigen} Mengen $M$ heißt Metrik, wenn sie (i)~$d(x,y)\geq0$, (ii)~$d(x,y)=0 \iff x=y$, (iii)~$d(x,y)=d(y,x)$ und (iv)~$d(x,y) \leq d(x,z)+d(y,z)$ erfüllt.
Ist $d$ auf einer Menge $M$ definiert, so heißt $(M,d)$ metrischer Raum; eine auf diesem Raum definierte Abbildung $f : M \to M$ heißt Isometrie.]

\subsection{\label{sub:Errechnung-von-Normalen}Errechnung von Normalen}

Sei $\vec{v}$ ein Vektor mit $n$ Komponenten, so lassen sich dazu $n-1$ senkrechte linear unabhängige Vektoren nach folgendem Prinzip errechnen:
\begin{enumerate}
  \item Auswahl einer Komponente $n_i \neq 0$.
  \item Vertauschung von $n_i$ mit einer anderen Komponente $n_j$ ($i\neq j$) und Setzung der restlichen Komponenten auf~$0$.
  \item Negation einer der gewählten gebliebenen Komponenten $n_i \neq 0$ oder $n_j \neq 0$.
\end{enumerate}

\subsection[Orthonormalisierungsverfahren]{Orthonormalisierungsverfahren nach \protect\noun{Gram-Schmidt}}

Wird benutzt, um aus $m$ linear unabhängigen Vektoren $\vec v_1, \ldots, \vec v_m$ paarweise orthogonale normalisierte Vektoren $\vec w_1, \ldots, \vec w_m$ zu erzeugen.

\begin{mathalgo}{Orthonormalisierung nach \protect\noun{Gram-Schmidt}}
$\vec w_1 \= \vec v_1 / \lVert\vec v_1\rVert$
for $k \= 1$ to $m-1$:
\> \cmt{Entfernen des nicht-orthogonalen Anteils aus $\vec v_{k+1}$}
\> $\vec r_{k+1} \= \vec v_{k+1} - \sum_{i \leq k} \langle \vec v_{k+1}, \vec w_i \rangle \vec w_i$
\smallskip
\> \cmt{Normalisierung von $\vec r_{k+1}$}
\> $\vec w_{k+1} \= \vec r_{k+1} / \lVert \vec r_{k+1} \rVert$
\end{mathalgo} 


\section{Operationen mit zwei Vektoren}
\begin{description}
  \item [Projektion von $\vec a$ in Richtung $\vec b$]
	\emph{Auch}: Komponente von $\vec a$ entlang $\vec b$; $\bigl(\langle \vec{a},\vec{b}\rangle \bigm/ \langle \vec{b},\vec{b}\rangle \bigr) \cdot \vec b$.
  \item [Winkel zwischen $\vec a$ und $\vec b$]
	\quad$\langle \vec{a},\vec{b}\rangle =\lVert\vec{a}\rVert\cdot\lVert\vec{b}\rVert\cdot\cos(\varphi)$
  \item [\noun{Schwarz}sche Ungleichung]
	$\lvert\langle \vec{a},\vec{b}\rangle\rvert \leq \lVert\vec{a}\rVert\cdot\lVert\vec{b}\rVert$
  \item [Dreiecksungleichung]
	$\lVert\vec a + \vec b \rVert \leq \lVert\vec a \rVert + \lVert \vec b \rVert$
\end{description}

\subsection{Kreuzprodukt\label{subsec:Kreuzprodukt}}

Das Kreuzprodukt (oder \emph{Vektorprodukt}) ist für $\mathbb{R}^{3}$ definiert als:
\begin{align*}
  \veccc(a_1;a_2;a_3)
  \times
  \veccc(b_1;b_2;b_3)
  &
  =\veccc(
    a_2 b_3 - a_3 b_2;
    a_3 b_1 - a_1 b_3;
    a_1 b_2 - a_2 b_1)
  \\
  \lVert \vec{a}\times\vec{b} \rVert
  &
  =\lVert\vec{a}\rVert \cdot \lVert\vec{b}\rVert \cdot \sin(\varphi)
\end{align*}


Es gelten folgende Eigenschaften:
\begin{description}
  \item [{Antikommutativität}]
	$\vec{a}\times\vec{b} = -(\vec{b}\times\vec{a})$
  \item [{Distributivität}]
	$\vec{a}\times(\vec{b}+\vec{c}) = (\vec{a}\times\vec{b}) + (\vec{a}\times\vec{c})$
	[Analog: $(\vec{a} + \vec{b}) \times \vec{c}$.]
  \item [{Skalierbarkeit}]
	$\forall\alpha\in\mathbb{R}
	  \ :\ 
	  (\alpha\vec{a}) \times \vec{b} = \alpha(\vec{a}\times\vec{b}) = \vec{a}\times(\alpha\vec{b})$
  \item [{Keine~Assoziativität}]
	In der Regel gilt nicht: $(\vec{a}\times\vec{b}) \times \vec{c} = \vec{a}\times(\vec{b}\times\vec{c})$
\end{description}
Außerdem gilt:
$\lVert \vec{a}\times\vec{b} \rVert^2 =
\langle \vec{a},\vec{a} \rangle
\cdot
\langle \vec{b},\vec{b} \rangle
-
\langle \vec{a},\vec{b} \rangle ^2$

\begin{figure}[H]
\centering\includegraphics{kreuzprodukt.pdf}

\caption{Kreuzprodukt}
\end{figure}

\index{Vektor|)}


\section{Vektorraum aus Polynomen}
\begin{description}
  \item [{Rekursive~Faktorzerlegung}] \index{Faktorzerlegung|see{Polynom}}\index{Polynom!Faktorzerlegung}
    $p_n(x) = (x-x_0) p_{n-1}(x) + r$
  \item [{Zerlegung~mit~$n$~Nullstellen}] \index{Nullstellen|see{Polynom}}\index{Polynom!Nullstellen}
    $p_n(x) = a_n\sum_{i=1}^n (x-x_i)$, wobei $x_i$ die Nullstellen sind.
    [Siehe \noun{Horner}schema auf Seite~\pageref{hornerschema}.]
\end{description}
\index{Basis|see{Vektorraum}}\index{Vektorraum!Basis}
Die Funktionen $p_n = \{1,x,x^2,\dots,x^n\}$ bilden eine Basis von $P_n$ (Polynome\index{Polynom} vom Grad $n$ oder kleiner): $\dim(P_n)=n+1$; $p_n$ ist genau dann linear unabhängig, wenn $\sum_i \lambda_i x^i = 0 \: \Rightarrow \: \lambda_{1,\ldots,n} = 0$ gilt.


\section{Ebenengleichungen}

\index{Ebene|(}
Sei $V$ ein $n$-dimensionaler Vektorraum über einem Körper $K \in \{\mathbb R, \mathbb C\}$, $\vec x$, $\vec p$, $\vec n$ sowie $\vec r_{1,\ldots,n-1} \in V$ und $\lambda_{1,\ldots,n-1} \in K$; $\vec p$ ist der Ortsvektor der (Hyper-)Ebene, $\vec r_{1,\ldots,n-1}$ sind die Richtungsvektoren, $\vec n$ ist die Normale.
Ein Körper der Dimension $n-1$ in $V$ wird als Hyperebene bezeichnet, für $n=2$ als Gerade und für $n=3$ einfach als Ebene.

\begin{description}
  \item [Parameterform] \index{Punkt-Richtungs-Form}\index{Ebene!Punkt-Richtungs-Form}\index{Ebene!Parameterform}\index{Parameterform}
	$E = \bigl\{ \vec x \bigm| \vec p + \sum_{1 \leq i < n} \lambda_i\vec r_i \bigr\}$.
	\emph{Auch}: Punkt-Richtungs-Form.
  \item [Normalform] \index{parameterlose Form}\index{Ebene!parameterlose Form}\index{Ebene!Normalform}\index{Normalform}
	$E = \bigl\{ \vec x \bigm| \langle \vec x-\vec p, \vec n \rangle = 0 \bigr\}$
	bzw.~$E = \bigl\{ \vec x \bigm| \langle \vec x, \vec n \rangle = \langle \vec p, \vec n \rangle \bigr\} $.
	\emph{Auch}: parameterlose Form.
  \item [\noun{Hesse}sche~Normalform] \index{Ebene!Hessesche Normalform}\index{Hessesche Normalform}\index{Normalform!Hessesche}
	Wie gewöhnliche Normalform, nur mit $\lVert\vec n\rVert=1$.
  \item [Koordinatenform] \index{Ebene!Koordinatenform}\index{Koordinatenform}
	Die Gleichung $\langle \vec{x},\vec{n}\rangle =\langle \vec{p},\vec{n}\rangle$ ausgeschrieben:
	\[ E = \bigl\{ \vec x \bigm| x_1 n_1 + \cdots + x_n n_n = \langle \vec{p},\vec{n}\rangle \bigr\} \]
\end{description}


\subsection{Umformungen}

\index{Ebene!Umformungen}

\begin{table}[h]
\centering
\begin{tabular}{lcccc}
  && \multicolumn{3}{c}{\emph{von}} \\
  && \bfseries PF & \bfseries NF & \bfseries KF \\
  \multirow{3}{*}{\begin{sideways} \emph{nach} \end{sideways}}
    & \bfseries PF & --- & via KF & s.\,u. \\
    & \bfseries NF & s.\,u. & --- & s.\,u. \\
    & \bfseries KF & via NF & s.\,u. & --- \\
\end{tabular}
\caption{Umformungswege Ebenengleichungen}
\end{table}



\subsubsection*{Parameterform zur \protect\noun{Hesse}schen Normalform}

Im $\mathbb{R}^{3}$:
\begin{enumerate}
  \item Normale $\vec{n}$ bestimmen: Vektorprodukt der Richtungsvektoren bilden und normalisieren.
  \item Gleichung aufstellen: $\langle \vec{x},\vec{n}\rangle =\langle \vec{p},\vec{n}\rangle $, dabei die rechte Seite durch das Ergebnis des Skalarprodukts ersetzen.
\end{enumerate}

\subsubsection*{Normalform zur Koordinatenform}

Die Gleichung $\langle \vec{x},\vec{n}\rangle =\langle \vec{p},\vec{n}\rangle $ ausgeschrieben:
$\sum_{1 \leq i < n} x_i n_i = \langle \vec p, \vec n \rangle $.

\subsubsection*{Koordinatenform zur \noun{Hesse}schen Normalform}
\begin{enumerate}
  \item Aus den Faktoren der $x_i$ den Normalenvektor $\vec{n}$ bestimmen (der $i$-te Faktor ist die $i$-te Komponente der Normale) und ggf.~normalisieren.
  \item Einen beliebigen Punkt $\vec{p}$ der Ebene ermitteln.
  \item Mithilfe der Parameterform die \noun{Hesse}sche Normalform ermitteln.
\end{enumerate}

\subsubsection*{Koordinatenform zur Parameterform}
\begin{enumerate}
  \item $n$ paarweise verschiedene Punkte der Ebene ermitteln.
  \item Einen Punkt als Ortsvektor $\vec{p}$ wählen.
  \item Von den restlichen Punkten $\vec{p}$ abziehen, das sind dann die Richtungsvektoren $\vec{r}_{1,\ldots,n-1}$.
\end{enumerate}

\subsubsection*{Alternativen}

Die Umrechnung der Koordinaten- und Normalform zur Parameterform lässt sich auch mit Hilfe der Errechnung von linear unabhängigen Normalen zur Ebenennormale durchführen.
Siehe dazu Seite \pageref{sub:Errechnung-von-Normalen}.

\index{Ebene|)}


\section{Abstandbestimmung}

\index{Abstand}\index{Vektor!Abstand}

Allgemein:
\begin{enumerate}
  \item Richtung $\vec r$ des Abstandsvektors bestimmen.
        [Bei zwei Geraden ist dies z.\,B.~das Kreuzprodukt beider Richtungsvektoren, bei einer Ebene und einer Geraden die Normale der Ebene, etc\@.]
  \item Vektor $\vec a$ als Vektor zweier beliebiger Punkte auf den beiden Objekten bestimmen.
        [Dies ist der sog.~\emph{Verbindungsvektor}, der die beiden Objekte in beliebiger Richtung verbindet.]
  \item Kürzester Abstandsvektor ist dann die Projektion des Verbindungsvektors auf den Abstandsvektor:
	\[ \vec d = \langle \vec a, \vec r \rangle \bigm/ \langle \vec r, \vec r \rangle \cdot \vec r \]
\end{enumerate}

\begin{description}
  \item [{Hyperebenen}]
	$\vec{r}$ entspricht der Normale der Ebene.
  \item [{Geraden~im~$\mathbb{R}^{3}$}]
	Funktioniert nur bei nicht-parallelen Geraden, $\vec{r}$ entspricht dem Kreuzprodukt der Richtungsvektoren.
\end{description}

\pagebreak[2]
\section{Schnittmengenbestimmung}

\CheckedBox{} Beispiel vorhanden auf \cpageref{sec:bsp-Schnittmengenbestimmung}.
\begin{enumerate}
  \item \index{Schnittmenge}Umformung beider Gleichungen in jeweils die parameterlose Form und die Parameterform.
  \item Parameterlose Form in die Parameterform einsetzen und Wert eines Parameters berechnen.
  \item Erhaltenen Parameter in die Parametergleichung einsetzen.
\end{enumerate}

\chapter{Matrizen}

\index{Matrix|(}


\section{Definition}

Eine $m\times n$-Matrix besteht aus $m$ Zeilen und $n$ Spalten. Die Indizes der einzelnen Werte bestehen aus der Zeile gefolgt von der Spalte.
\[
A = \begin{pmatrix}
  a_{11} & a_{12} & \cdots & a_{1n}\\
  a_{21} & a_{22} & \cdots & a_{2n}\\
  \vdots & \vdots &        & \vdots\\
  a_{m1} & a_{m2} & \cdots & a_{mn}
\end{pmatrix}
\]

\index{Matrix!Typ}\index{Typ}
Die Elemente einer Matrix vom Typ $(m,n)$ lassen sich mit $a_{ij},\; 1\leq i\leq m \; \land \; 1\leq j\leq n$ ansprechen.
[Der Typ kann auch als $m\times n$ geschrieben werden.]
Man kann Matrizen auch alternativ $A=(a_{ij}),\, 1 \leq i \leq m,\, 1 \leq j \leq n$ schreiben.
Die Menge aller $m\times n$-Matrizen über einem Körper $K$ wird als $M(m\times n,K)$ bzw.~$K^{m\times n}$ geschrieben.

\begin{description}
  \item [{Gleichheit}]
	$A=B$ bzw.~$(a_{ij})=(b_{ij})$ gilt, wenn die Typen gleich sind und $\forall i,j:a_{ij}=b_{ij}$ gilt.
  \item [{Addition,~Subtraktion}]
	$A+B:=(a_{ij}+b_{ij})$
  \item [{Nullmatrix}] \index{Nullmatrix}\index{Matrix!Null--}
	$a_{ij}=0,\; 1 \leq i \leq m,\; 1 \leq j \leq n$
  \item [{Unipotent}] \index{Matrix!unipotent}\index{unipotent}
	$a_{ii}=1,\; 1 \leq i \leq n$
  \item [{Einheitsmatrix}] \index{Einheitsmatrix}\index{Matrix!Einheits--}\index{Matrix!Kronecker-Delta}\index{Kronecker-Delta}
	$E=(\delta_{ij})$ mit dem \noun{Kronecker}-Delta
	\[
	  \delta_{ij}=\begin{cases}
	    0 & \hbox{falls $i\neq j$}\\
	    1 & \hbox{falls $i=j$}
	  \end{cases}
	\]
  \item [{Transponierte~Matrix}] \index{transponierte Matrix}\index{Matrix!transponiert}
	Die transponierte Matrix $A^T$ einer Matrix $A$ erhält man durch Vertauschung von Zeilen und Spalten.
  \item [{Inverse~Matrix}] \index{inverse Matrix}\index{Matrix!invers}
	Für die inverse Matrix $A^{-1}$ einer Matrix $A \in K^{n \times n}$ gilt $AA^{-1}=A^{-1}A=E$. Es gibt nicht immer ein $A^{-1}$.
  \item [{Quadratisch}] \index{quadratische Matrix}\index{Matrix!quadratisch}
	$m=n$
  \item [{Zeilenmatrix}] \index{Matrix!Zeilen--}\index{Zeilenmatrix}
	$m=1$
  \item [{Spaltenmatrix}] \index{Matrix!Spalten--}\index{Spaltenmatrix}
	$n=1$
  \item [{Rang}] \index{Rang}\index{Matrix!Rang}
	Der Rang einer Matrix ist die Dimension der größten Untermatrix $U$, für die $\det(U)\neq0$ ist; dies entspricht der Dimension der Spalten- bzw.~Zeilenvektoren.

    [Eine Möglichkeit zur Bestimmung des Ranges ist die Anwendung des \noun{Gauß}-Verfahrens für lineare Gleichungssysteme und die Anzahl der nicht verschwindenden Zeilen zu zählen.]
\end{description}

\section{Rechenregeln}
\begin{description}
  \item [{Multiplikation~mit~Skalar}]
	$\lambda A:=(\lambda a_{ij})$
  \item [{Multiplikation~(allgemein)}]
	Voraussetzung: $A$ ist eine $m\times n$-Matrix, $B$ eine $n\times r$-Matrix.
	Dann gilt: $C=A\cdot B:=(c_{ik})$ mit $c_{ik}=\sum a_{ij} b_{jk}$.
	Es entsteht eine $m\times r$-Matrix.
	\emph{Achtung:} Im allgemeinen ist $AB\neq BA$.
\end{description}

Sofern die Produkte existieren, gelten:
\begin{description}
  \item [{Distributivgesetz}] \index{Matrix!Distributivgesetz}
	$A(B+C)=AB+AC$
  \item [{Assoziativgesetz}] \index{Matrix!Assoziativgesetz}
	$(AB)C=A(BC)=ABC$
\end{description}
\[
    (AB)\T=B\T A\T \qquad EA=AE=A
\]

\subsection{\protect\noun{Falk}sches Schema}
\index{Falksches Schema}\index{Matrix!Falksches Schema}\index{Matrix!Multiplikation}

Das \noun{Falk}sche Schema kann als Hilfsmittel zur Matrizenmultiplikation benutzt werden.

Hier ein Beispiel für 
\[
\begin{pmatrix}
  1 & 4\\
  2 & 5\\
  3 & 6
\end{pmatrix}
\cdot
\begin{pmatrix}
  1 & 2\\
  0 & 1
\end{pmatrix}
\]

\Todo{Darstellung Falksches Schema reparieren}

\vbox{
\emph{Schritt 1:} Die rechte Matrix etwas höher aufschreiben als die linke.
\[
\xySep{5pt}\xymatrix{
                &  & \,\ar@{-}[ddddd] & 1 & 2 \\
                &  &                  & 0 & 1 \\
\,\ar@{-}[rrrr] &  &                  &   & \,\\
1               & 4\\
2               & 5\\
3               & 6                   & \,
}
\]
}

\bigskip{}

\vbox{
\emph{Schritt 2:} Quasi die Skalarprodukte aus den Zeilen von $A$ und den Spalten von $B$ bilden.
\[
\xySep{5pt}\xymatrix{
                &   & \,\ar@{-}[ddddd] & 1\ar@{..>}[ddd] & 2\\
                &   &                  & 0               & 1\\
\,\ar@{-}[rrrr] &   &                  &                 & \,\\
1\ar@{..>}[rrr] & 4 &                  & {\color{red}1}\\
2               & 5\\
3               & 6 & \,
}
%
\quad
%
\xySep{5pt}
\xymatrix{
                &   & \,\ar@{-}[ddddd] & 1\ar@{..}[ddddd] & 2 \\
                &   &                  & 0                & 1 \\
\,\ar@{-}[rrrr] &   &                  &                  & \,\\
1               & 4 &                  & 1                    \\
2\ar@{..}[rrr]  & 5 &                  & {\color{red}2}       \\
3\ar@{..}[rrr]  & 6 & \,               & {\color{red}3}
}
%
\quad
%
\xySep{5pt}
\xymatrix{
                &   & \,\ar@{-}[ddddd] & 1 & 2\ar@{..}[ddddd]\\
                &   &                  & 0 & 1               \\
\,\ar@{-}[rrrr] &   &                  &   & \,              \\
1\ar@{..}[rrrr] & 4 &                  & 1 & {\color{red}6}  \\
2\ar@{..}[rrrr] & 5 &                  & 2 & {\color{red}9}  \\
3\ar@{..}[rrrr] & 6 & \,               & 3 & {\color{red}12}
}
\]
}

\bigskip{}

\vbox{
\emph{Schritt 3:} Produkt abschreiben.
\[
C = \begin{pmatrix}
  1 & 6\\
  2 & 9\\
  3 & 12
\end{pmatrix}
\]
}


\section{Quadratische Matrizen}
\begin{description}
  \item [{Hauptdiagonale}] \index{Matrix!Diagonale}\index{Diagonale!Haupt--}\index{Hauptdiagonale}\index{Matrix!Hauptdiagonale}\index{Matrix!quadratisch}
	existiert nur bei quadratischen Matrizen ($m=n$), es sind die Elemente $(a_{kk}),\; 1\leq k\leq n$ gemeint.
  \item [{Nebendiagonalen}] \index{Diagonale!Neben--}\index{Nebendiagonale}\index{Matrix!Nebendiagonale}
	Siehe unten.
	Nebendiagonalen (markiert mit $n$) heißen sowohl die Diagonalen direkt über und direkt unter der Hauptdiagonalen ($H$) als auch die zur Hauptdiagonalen senkrechte Diagonale.
\end{description}
\Todo{Darstellung Haupt-/Nebendiagonalen reparieren}
\[
  \xySep{5pt}
  \left(\vcenter{\xymatrix{
  H\ar[ddddrrrr]    & n\ar@{.>}[dddrrr] & \circ & \circ & n\ar@{.>}[ddddllll] \\
  n\ar@{.>}[dddrrr] & \circ             & \circ & \circ & \circ               \\
  \circ             & \circ             & \circ & \circ & \circ               \\
  \circ             & \circ             & \circ & \circ & n                   \\
  n                 & \circ             & \circ & n     & H
  }
  }\right)
\]

\begin{description}
  \item [{Dreiecksmatrix}] \index{Matrix!Dreiecks--}\index{Dreiecksmatrix}
	Wenn alle Elemente über der Hauptdiagonalen 0 sind heißt die Matrix \enquote{untere Dreiecksmatrix}.
	Sind die Elemente \emph{unter} der Hauptdiagonalen 0, so heißt die Matrix \enquote{obere Dreiecksmatrix}.
  \item [{Strikte~Dreiecksmatrix}] \index{Matrix!Dreiecksmatrix!strikt}\index{Strikte Dreiecksmatrix}
	Ist bei einer Dreiecksmatrix die Hauptdiagonale 0, so heißt die Matrix \emph{strikte} Dreiecksmatrix.
  \item [{Diagonalmatrix}] \index{Matrix!Diagonal--}\index{Diagonalmatrix}
	$i\neq j\Rightarrow a_{ij}=0$
  \item [{Symmetrische~Matrix}] \index{Matrix!symmetrisch}\index{symmetrische Matrix}
	$a_{ij}=a_{ji} \iff A=A^{T}$
  \item [{Tridiagonalmatrix}] \index{Matrix!Tridiagonal--}\index{Tridiagonalmatrix}
    Eine Diagonalmatrix, bei der die Nebendiagonalen direkt neben der Hauptdiagonalen von 0 verschieden sein dürfen.
\end{description}

\section{\label{sec:Lineare-Gleichungssysteme}Lineare Gleichungssysteme}

\index{Gleichungssystem!linear}\index{lineares Gleichungssystem}
Ein \gls{lgs} besteht aus $m$ linearen Gleichungen und $n$ Unbekannten:
\[
\begin{matrix}
  a_{11} x_1 & +      & a_{12} x_2 & +      & \ldots & +      & a_{1n} x_n & =      & b_1   \\
  a_{21} x_1 & +      & a_{22} x_2 & +      & \ldots & +      & a_{2n} x_n & =      & b_2   \\
  \vdots     & \vdots & \vdots     & \vdots & \vdots & \vdots & \vdots     & \vdots & \vdots\\
  a_{m1} x_1 & +      & a_{m2} x_2 & +      & \ldots & +      & a_{mn} x_n & =      & b_m
\end{matrix}
\]

\begin{description}
  \item [{Homogen}] \index{Gleichungssystem!linear!homogen}\index{homogen}
	$\forall i:b_i=0$
  \item [{Inhomogen}] \index{Gleichungssystem!linear!inhomogen}\index{inhomogen}
	$\exists i:b_i\neq0$
  \item [{Koeffizientenmatrix}] \index{Matrix!Koeffizienten--}\index{Koeffizientenmatrix}
	$(a_{ij})$
  \item [{Erweiterte~Matrix}] \index{Matrix!erweiterte}\index{erweiterte Matrix}
	Koeffizientenmatrix erweitert um die $b_i$
  \item [{Pivot-Element}] \index{Matrix!Pivot-Element}\index{Pivot-Element}
	Das erste Element einer Zeile $\neq0$.
	[Auch die Lösungsspalte kann ein Pivot-Element enthalten.]
  \item [{Pivot-Spalte}] \index{Matrix!Pivot-Spalte}\index{Pivot-Spalte}
	Spalte mit Pivot-Element.
  \item [{Äquivalenz}] \index{Matrix!Äquivalenz}
	Zwei Matrizen $A,B$ sind äquivalent (Schreibweise: $A\sim B$), wenn $A^{-1}\vec b = B^{-1}\vec b$ für ein beliebiges, aber festes $\vec b$ gilt.
	[Die Matrizen besitzen die gleichen Lösungen für ein gegebenes $\vec b$.]
  \item [{Stufenform}] \index{Matrix!Stufenform}\index{Stufenform}
	Die Koeffizientenmatrix der erweiterten Matrix bildet eine obere Dreiecksmatrix (von oben nach unten verschiebt sich die Pivot-Spalte nach rechts).
  \item [{Reduzierte~Stufenform}] \index{Matrix!Stufenform!reduziert}\index{reduzierte Stufenform}\index{Stufenform!reduziert}
	Die Koeffizientenmatrix hat pro Zeile und Pi\-vot-Spal\-te genau eine 1 und sonst nur 0.
	Bei quadratischen Matrizen bildet die Koeffizientenmatrix eine Einheitsmatrix.
  \item [{Unterbestimmt}] \index{unterbestimmt}\index{Gleichungssystem!unterbestimmt}
	Weniger linear unabhängige Gleichungen als Unbekannte.
    
	[Für $A \in K^{m\times n}$ ist $m<n$ und $\rg(A) \leq m$.]
  \item [{Überbestimmt}] \index{überbestimmt}\index{Gleichungssystem!überbestimmt}
	Mehr linear unabhängige Gleichungen als Unbekannte.
    
	[Für $A \in K^{m\times n}$ ist $m>n$ und $\rg(A) \leq n$.]
  \item [{Nicht~lösbar}]
	Lösungsspalte der reduzierten Stufenform ist eine Pivot-Spalte.
  \item [{Eindeutig~lösbar}]
	Die Koeffizientenmatrix der reduzierten Stufenform ist eine Einheitsmatrix. Bei quadratischen
	Koeffizientenmatrizen ist die Determinante $\neq0$ (Gleichungen sind linear unabhängig).
  \item [{Mehrdeutig~lösbar}]
	Das \gls{lgs} ist unterbestimmt. Bei quadratischen Koeffizientenmatrizen ist die Determinante $=0$ (Gleichungen sind linear abhängig).
\end{description}

\subsection{Lösungsmethoden}

Einfache Verfahren sind Gleichsetzungs-, Einsetzungs- und Additionsverfahren (sinnvoll bis max.~3 Unbekannte).
\begin{description}
  \item [{\noun{\noun{Gauß}}-Algorithmus}]
	Standardverfahren, auch bekannt als \noun{Gauß}sches Eliminationsverfahren.
  \item [{\noun{Cramer}sche~Regel}]
	Lösungsverfahren mit Hilfe von Determinanten.
  \item [{Methode~der~kleinsten~Quadrate}]
	Wird benutzt, um bei überbestimmten Gleichungssystemen $A$ diejenige Lösung $\vec x^*$ zu ermitteln, für die $\lVert A\vec x^* - \vec b \rVert_2$ minimal ist.
  \item [{Verallgemeinerte~Inverse}]
	Wird bei unterbestimmten Gleichungssystemen benutzt, um die Lösung $\vec x^*$ mit der geringsten Norm $\lVert \vec x^* \rVert$ zu ermitteln.
\ifincludeincomplete
  \item [{\noun{Cholesky}-Zerlegung}]
	Verfahren der Numerik für symmetrische, positiv-de\-fi\-ni\-te Matrizen, siehe Seite \pageref{sec:Cholesky-Zerlegung}.
\fi
\end{description}

\subsubsection{\protect\noun{Gauß}-Algorithmus}

\index{Algorithmus!Gauß}\index{Gauß-Algorithmus}
\begin{description}
  \item [{Ziel}] Bestimmung der zu einem \gls{lgs} äquivalenten reduzierten Stufenform unter Verwendung von Äquivalenzumformungen.
\end{description}

\paragraph*{Verwendete Rechenschritte}
\begin{itemize}
  \item Addition eines Vielfachen einer Zeile zu einer anderen
  \item Vertauschen zweier Zeilen
  \item Skalarmultiplikation mit einem Skalar $\lambda\in\mathbb{R}\setminus\{0\}$
\end{itemize}

\paragraph*{Algorithmus}
\begin{enumerate}
  \item Spaltenweises Vorgehen: \Gls{lgs} in eine Stufenform überführen. [Die finale Koeffizientenmatrix ist nicht eindeutig.]
  \item Zeilenweises Vorgehen: Von unten nach oben das Pivotelement auf 1 bringen, Elemente darüber auf 0 bringen.
\end{enumerate}

\begin{mathalgo}{\protect\noun{Gauß}-Algorithmus}
for $i \= 1$ to $n-1$:
\> for $j \= i+1$ to $n$:
\>\> $a_{j,1\ldots n} \= a_{j,1\ldots n} - (a_{ji} / a_{ii}) \cdot a_{i,1\ldots n}$
\end{mathalgo}


\subsubsection{\protect\noun{Cramer}sche Regel}

\index{Regel!Cramersche}\index{Cramersche Regel}
Sei $A\vec{x}=\vec{b}$ ein lineares Gleichungssystem mit $A\in M(n\times n,K)$ und $\vec{x}, \vec{b} \in K^n$ sowie $\det(A)\neq0$, dann wird eine Matrix $A_i$ definiert als Matrix $A$, in der die $i$-te Spalte durch $\vec{b}$ ersetzt wurde.Die Lösung des Gleichungssystems ist dann gegeben als $x_i = \det(A_i) / \det(A), 1 \leq i \leq n$.


\subsubsection{\label{sub:Methode-der-kleinsten-Quad}Methode der kleinsten Quadrate}

\index{Methode der kleinste Quadrate}
Sei $A\in M(m\times n,\mathbb{R})$, $m\geq n$, $\rg(A)=n$, $A\vec{x}=\vec{b}$, $\vec{b}\in\mathbb{R}^m$, dann gilt für die beste Näherungslösung:
\[ \vec{x}=(A\T A)^{-1} A^T \vec{b} \]



\subsubsection{Verallgemeinerte Inverse}

\index{Inverse!verallgemeiner}\index{verallgemeinerte Inverse}
Sei $A\in M(m\times n,\mathbb{R})$, $m<n$, $\rg(A)=m$, $A\vec{x}=\vec{b}$, $\vec{b}\in\mathbb{R}^{m}$, dann gilt für die beste Näherungslösung mit der kleinsten Norm von $\vec{x}$:
\[ \vec{x} = A^T (AA^T)^{-1} \vec{b} \]



\subsubsection{Darstellung unendlich vieler Lösungen}
\CheckedBox{} Beispiel vorhanden auf \cpageref{sec:bsp-LGS-unendlich}.

\begin{enumerate}
  \item Reduzierte Stufenform errechnen.
  \item Nicht-Pivot-Spalten mit dem entsprechenden freien Koeffizienten durch Subtraktion auf die rechte Seite bringen.
  \item Erweiterung um die fehlenden Zeilen: Konstante Vektoren um 0 er\-gänz\-en, abhängige Vektoren der entsprechenden freien Koeffizienten um 1 er\-gänz\-en.
\end{enumerate}


\section{Determinanten}

\index{Matrix!Determinante}\index{Determinante}
Determinanten können nur für quadratische Matrizen berechnet werden.
Ihr Wert ist ein Skalar.
Bis zur Größe $3\times3$ gibt es spezielle Gleichungen.
\[ \det(a_1)=a_1 \]
\[
	\det\begin{pmatrix}
	  a_1 & b_1\\
	  a_2 & b_2
	\end{pmatrix}
	=
	\det(\vec{a},\vec{b}) = a_1 b_2 - b_1 a_2
\]
\begin{align*}
\det\begin{pmatrix}
 a_1 & b_1 & c_1\\
 a_2 & b_2 & c_2\\
 a_3 & b_3 & c_3
\end{pmatrix}
& = \det(\vec{a},\vec{b},\vec{c}) \\
& = a_1 b_2 c_3 + b_1 c_2 a_3 + c_1 a_2 b_3 \\
& - a_1 c_2 b_3 - b_1 a_2 c_3 - c_1 b_2 a_3
\end{align*}

Letzteres lässt sich mit der Regel von \noun{Sarrus} veranschaulichen:
\[
\xySep{5pt}\xymatrix{
        & a_1\ar@{.>}[dddrrr] & b_1\ar@{.>}[dddrrr] & c_1\ar@{.>}[dddlll]\ar@{.>}[dddrrr] & a_1\ar@{.>}[dddlll] & b_1\ar@{.>}[dddlll]\\
        & a_2                 & b_2                 & c_2                                 & a_2                 & b_2\\
        & a_3                 & b_3                 & c_3                                 & a_3                 & b_3\\
\color{red}\ominus & \color{red}\ominus & \color{red}\ominus & & \color{green}\oplus & \color{green}\oplus & \color{green}\oplus
}
\]

\subsection{Rechenregeln}

Die Regeln für die Spalten gelten Aufgrund von D6 auch für Zeilen.
\begin{description}
  \item [{D1}]
	$\det(\vec a, \vec b, \ldots, \vec z) = \det(\vec z, \vec a, \vec b, \ldots) = \det(\vec y, \vec z, \vec a, \ldots)$
	[Rotation von Spalten]
  \item [{D2}]
	$\det(\vec a, \vec b, \vec c, \ldots) = -\det(\vec b, \vec a, \vec c, \ldots)$
	[Vertauschung von Spalten]
  \item [{D3}]
	$\det(\vec a, \vec a, \vec b, \ldots) = 0$
	[Duplikat einer Spalte]
  \item [{D4}]
	$\forall\alpha\in\mathbb R : \det(\alpha\cdot\vec a, \vec b, \vec c, \ldots) = \alpha\cdot\det(\vec a, \vec b, \vec c, \ldots)$
	[Multiplikation einer Spalte mit einem Skalar]
  \item [{D5}]
	$\det(\vec a, \vec b+\vec b', \vec c, \ldots) = \det(\vec a, \vec b, \vec c, \ldots) + \det(\vec a, \vec b', \vec c, \ldots)$
	[Addition zweier Spalten]
  \item [{D6}]
	$\det(A)=\det(A^T)$
	[Invarianz bei Transponierung]
\end{description}

Außerdem gilt im $\mathbb{R}^{3}$:
\[ \langle \vec{a}\times\vec{b},\vec{c}\rangle = \det(\vec{a},\vec{b},\vec{c}) \]
[Die Determinante im $\mathbb{R}^{3}$ gibt also das Volumen des durch die Vektoren $\vec{a}$, $\vec{b}$ und $\vec{c}$ aufgespannten Spats an.]

Weiter gilt: $\det(AB)=\det(A)\cdot\det(B)$.


\subsection{Entwicklungssatz nach \protect\noun{Laplace}}

\index{Laplace!Entwicklungssatz}\index{Entwicklungssatz nach Laplace}
Sei $A_{ij}$ definiert als die $(n-1)\times(n-1)$-Untermatrix der Matrix $A$ durch Streichen von Zeile $i$ und Spalte $j$.
Dann gilt für die Determinante:
\begin{description}
  \item [{Entwicklung~nach~Zeile~$i$}]  $\quad\det(A)=\sum_{1 \leq j \leq n} (-1)^{i+j}\cdot a_{ij}\cdot\det(A_{ij})$
  \item [{Entwicklung~nach~Spalte~$j$}] $\quad\det(A)=\sum_{1 \leq i \leq n} (-1)^{i+j}\cdot a_{ij}\cdot\det(A_{ij})$
\end{description}
Dabei ergibt $(-1)^{i+j}$ ein Schachbrettmuster:
\[
\begin{pmatrix}
  + & - & + & \cdots\\
  - & + & - & \cdots\\
  + & - & + & \cdots\\
  \vdots & \vdots & \vdots & \ddots
\end{pmatrix}
\]



\subsection{Berechnung nach \protect\noun{Gauß}}

\index{Gauß!Determinante}\index{Determinante!Gauß}
Formt man die Matrix in eine Dreiecksmatrix um, so gilt für die Determinante:
\[ \det(A) = \prod a_{ii} = a_{11} \cdot a_{22} \cdots a_{nn} \]



\subsection{Komplementärmatrix}

\index{Komplementärmatrix}\index{Adjunkte}\index{Determinante!Minor}\index{Minor}
Die Komplementärmatrix (auch \emph{Adjunkte} genannt) $\tilde{A}=(\tilde{a}_{ij})$ zu einer Matrix $A$ ist mit Hilfe der Unterdeterminanten (\emph{Minore}, Einzahl \emph{Minor}) definiert als
\begin{align*}
  \tilde{a}_{ij} & =(-1)^{i+j}\cdot\det(A_{ji})\\
  \tilde{A}      & =\adj(A)
\end{align*}


Es bietet sich an:
\begin{enumerate}
  \item Aufschreiben der Minoren-Determinanten und nach dem Schachbrettmuster negieren (also $\tilde A\t$ aufschreiben).
  \item Transponieren der neuen Matrix.
\end{enumerate}

\subsection{Umkehrmatrix}

\index{Umkehrmatrix}\index{Matrix!Umkehr--}


\subsubsection[\protect\noun{Gauß}-\protect\noun{Jordan}-Algorithmus]{Berechnung mit dem \protect\noun{Gauß}-\protect\noun{Jordan}-Algorithmus}

\index{Algorithmus!Gauß-Jordan}\index{Jordan!Gauß-Jordan-Algorithmus}\index{Gauß-Jordan-Algorithmus}
Funktioniert wie die Lösung eines linearen Gleichungssystems.
Man Schreibt eine Blockmatrix der Form $A|E$ auf und versucht, diese mit den Rechenregeln für Determinanten in die Form $E|B$ zu bringen.
Gelingt dies, so ist $B=A^{-1}$.


\subsubsection[Komplementärmatrix]{Berechnung mit der Komplementärmatrix}

Es gilt \[ A^{-1} = \frac{1}{\det(A)}\tilde{A} \]


\subsection{Definitheit}

\index{Definitheit}\index{Matrix!Definitheit}
Eine symmetrische Matrix $A$ ist:
\[
  \forall\vec{x}\neq\vec{0}:\vec{x}^{T}\cdot A\cdot\vec{x} \begin{cases}
  	>    0       & \hbox{positiv definit}      \\
  	\geq 0       & \hbox{positiv semi-definit} \\
  	<    0       & \hbox{negativ definit}      \\
  	\leq 0       & \hbox{negativ semi-definit} \\
  	\hbox{sonst} & \hbox{indefinit}
  \end{cases}
\]

\subsubsection{\label{sub:Hurwitz-Kriterium}\protect\noun{Hurwitz}-Kriterium}

\index{Hurwitz-Kriterium}\index{Kriterium!Hurwitz}
Sei $D_k$ die Determinante der $k\times k$-Teilmatrix von $A \in \mathbb{R}^{n \times n}$ von links oben bzw.~rechts unten.
Dann gilt für $A$, $1 \leq k \leq n $:
\begin{center}
    \begin{tabular}{ll}
    	$D_k>0$           & positiv definit      \\
    	$D_k>0$           & positiv semi-definit \\
    	$(-1)^k D_k>0$    & negativ definit      \\
    	$(-1)^k D_k\geq0$ & negativ semi-definit
    \end{tabular}
\end{center}
\ifincludeincomplete
[Zur Überprüfung auf \emph{symmetrische} positive Definitheit bietet das \noun{Cholesky}-Verfahren auf \cpageref{sec:Cholesky-Zerlegung} eine einfachere Möglichkeit.]
\fi


\section{Transformationsmatrizen}

\index{Transformationsmatrix}\index{Matrix!Transformations--}
Die Koordinaten eines Vektors $\vec{x}$ zu einer Basis $A$ werden als $K_A(\vec{x})$ geschrieben und geben die Linearfaktoren zur Basis $A$ an, wodurch die Linearkombination der Vektoren von $A$ zu $\vec{x}$ eindeutig bestimmt ist.

Zur Transformation der Koordinaten von einer Basis $A$ zu einer Basis $B$ und umgekehrt gilt folgendes:
\begin{description}
  \item [{Von~$A$~nach~$B$}] $K_{B}(\vec{x})=\underbrace{B^{-1}A}_{T_{B}^{A}}\cdot K_{A}(\vec{x})$
  \item [{Von~$B$~nach~$A$}] $K_{A}(\vec{x})=\underbrace{A^{-1}B}_{T_{A}^{B}}\cdot K_{B}(\vec{x})$
\end{description}
Dabei heißt $T_{B}^{A}$ \emph{Transformationsmatrix} von $A$ nach $B$. In $T_{B}^{A}$ stehen die Koordinaten der Einheitsvektoren von $A$ zur Basis $B$.


\subsection{Basistransformation}

\index{Basistransformation}\index{Matrix!Basistransformation}
Seien $A$ und $A'$ Basen des $\mathbb{R}^{m}$ sowie $B$ und $B'$ Basen des $\mathbb{R}^{n}$ und $\Phi:\mathbb{R}^{m}\to\mathbb{R}^{n}$ eine lineare Abbildung, dann lässt sich eine \emph{Abbildungsmatrix} $M_{B}^{A}(\Phi)$ von $\Phi$ bezüglich $A$ und $B$ angeben:
\begin{align*}
  K_B(\Phi(\vec{x})) & = M_B^A(\Phi)\cdot K_A(\vec{x})\\
  \\
  M_{B'}^{A'}(\Phi)  & = T_{B'}^B\cdot M_B^A(\Phi)\cdot T_A^{A'}
\end{align*}
\[
  \xySep{12pt}\vcenter{\xymatrix{
    K_A(\vec{x})\ar@{->}[r]^{M_B^A}                                & K_B(\Phi(\vec{x}))\ar@{->}[d]^{T_{B'}^B}\\
    K_{A'}(\vec{x})\ar@{->}[u]_{T_A^{A'}}\ar@{->}[r]_{M_{B'}^{A'}} & K_{B'}(\Phi(\vec{x}))
  }}
\]



\section{\label{sec:Orthogonale-Matrizen}Orthogonale Matrizen}

\index{Matrix!orthogonal}\index{orthogonale Matrix}
Eine Matrix $A\in M(n\times n,\mathbb{R})$ heißt orthogonal, wenn $A\t A=E$ bzw.~$A\t=A^{-1}$ gilt.
Man schreibt $A\in O(n)$.
\begin{itemize}
  \item Die Spalten- bzw.~Zeilenvektoren von $A$ bilden jeweils ein Orthonormalsystem, daher gilt $\lvert\det(A)\rvert=1$.
  \item Abbildungen sind \emph{kongruent}, d.\,h., sie verändern nicht die Länge der abgebildeten Vektoren: $\lVert Ax \rVert_2 = \lVert x \rVert_2$.
	[Falls $\det(A)=1$, so handelt es sich um eine Drehung, falls $\det(A)=-1$, so handelt es sich um eine Spiegelung.]
  \item Das Produkt zweier Orthogonalmatrizen ist wieder eine Orthogonalmatrix.
  \item \index{orthogonaleMatrix!involut}\index{selbstinvers}\index{involut}\index{Matrix!involut}\index{Matrix!selbstinvers}
	Ist eine orthogonale Matrix symmetrisch, d.\,h.,~gilt $A^2=E$, so heißt $A$ \emph{involut} (selbstinvers).
\end{itemize}

\begin{figure}[htb]
\centering\includegraphics{matrix-spiegelung.pdf}\hfil\includegraphics{matrix-drehung.pdf}

\caption{Spiegelung und Drehung}
\end{figure}

\begin{description}
\item [{Drehmatrix~im~$\mathbb{R}^{2}$}]
\[
  A = \begin{pmatrix}
    \cos(\varphi) & -\sin(\varphi) \\
    \sin(\varphi) &  \cos(\varphi)
  \end{pmatrix}
\]
\end{description}

\section{Eigenwerte und -vektoren}

\index{Matrix!Eigenvektor}\index{Matrix!Eigenwert}\index{Eigenvektor}\index{Eigenwert}
\begin{enumerate}
  \item Lösung des \emph{charakteristischen Polynoms} $p(\lambda)=\det(A-\lambda E)\stackrel{!}{=}0$, die gefundenen $\lambda$ sind die \emph{Eigenwerte}.
  \item Einsetzen der $\lambda_{i}$ in $(\lambda_{i}E-A)\vec{x}=0$.
  		Die \emph{speziellen} Lösungen $\vec{x}_{i}$ sind die \emph{Eigenvektoren} zu den entsprechenden Eigenwerten.
\end{enumerate}
\begin{description}
  \item [{Algebraische~Vielfachheit}] Vielfachheit eines Eigenwertes.
  \item [{Geometrische~Vielfachheit}] Anzahl linear unabhängiger Eigenvektoren zu einem bestimmen Eigenwert.
\end{description}
\begin{itemize}
  \item Sind $A,T\in\mathbb{R}^{n\times n}$ mit $\det(T)\neq0$, so haben $A$ und $TAT^{-1}$ dieselben Eigenwerte.
  \item Hat $A\in\mathbb{R}^{n\times n}$ genau $n$ verschiedene Eigenwerte $\lambda_{i}$ und $n$ linear unabhängige Vektoren $v_{i}$, dann gilt:
	\[
	  A = T \cdot
	  \begin{pmatrix}
	    \lambda_1                     \\
		      & \ddots            \\
		      &        & \lambda_n
	  \end{pmatrix}
	  \cdot T^{-1}
	  ,\quad
	  T=(v_1, \ldots, v_n)
	\]
\end{itemize}

\subsection{\label{sub:Satz-von-Gerschgorin}Satz von \protect\noun{Gerschgorin}}

\index{Satz!von Gerschgorin}\index{Gerschgorin}
Sei $A\in\mathbb{R}^{n\times n}$, so gilt für die Eigenwerte $\lambda_{i}$:
\[ \lambda_{i}\in\bigcup \bigl\{ z\in\mathbb{C} \bigm| \lvert z-a_{ii} \rvert \leq \sum_{j\neq i} \lvert a_{ij} \rvert\bigr\} \]

[Dies ist eine Reihe von Kreisen in der komplexen Ebene mit Mittelpunkt bei $a_{kk}+0i$ und Radius $\sum_{j\neq i} \lvert a_{ij} \rvert$.]

\index{Matrix|)}


\chapter{\label{chap:lineare-abbildungen}Lineare Abbildungen}

\index{Homomorphismus}\index{Lineare Abbildung}\index{Abbildung!linear}


\section{Definitionen}

Seien $V$, $W$ Vektorräume über einem Körper $K$ ($f:V\to W$), dann heißt $f$ \emph{lineare Abbildung} bzw.~\emph{Homomorphismus},
wenn $\forall x,y\in V$ und $\forall\lambda\in K$ gilt:
\[ f(x+\lambda y)=f(x)+\lambda f(y) \]
Die Menge aller Homomorphismen über zwei Vektorräume $V$, $W$ wird als~$\hom(V,W)$ oder~$\mathcal L(V, W)$ bezeichnet.
\index{Dualraum}\index{Funktional}\index{Linearform}\index{1-Form}
[Die Menge aller linearen Abbildungen~$V^* := V \to K$ wird als Dualraum bezeichnet; die Elemente~$f \in V^*$ werden Funktionale, Linearformen oder 1-Formen genannt.]
\begin{description}
  \item [{Bild}] \index{Abbildung!Bild}\index{Bild}
	$\bild(f) = f(V) := \{ f(v) \mid v \in V \} \subseteq W$
	[$V$ ist Unterraum von $W$, wenn $f$ linear ist.]

	Die lineare Hülle der Spaltenvektoren der $m\times n$-Abbildungsmatrix einer Abbildung $f:\mathbb{R}^{n}\to\mathbb{R}^{m}$ ist gleich $\bild(f)$.
  \item [{Kern}] \index{Kern}\index{Abbildung!Kern}
	$\ker(f):=\{ v\in V \mid f(v)=0 \}$
    [Auch genannt: \emph{Nullraum} von $f$.]
    
    Es ist immer $0\in\ker(f)$.
	Außerdem: $\lvert \ker(f) \rvert \in \{ 1,\infty\} $
  \item [{Rang}] \index{Rang}\index{Abbildung!Rang}
	$\rg(f)=\dim(\bild(f))$
  \item [{Dimensionssatz}] \index{Abbildung!Dimensionssatz}\index{Dimensionssatz}\index{Satz!Dimensions--}
	$\dim(V)=\dim(\ker(f))+\rg(f)$.
    Wenn $\dim(V)=\dim(W)<\infty$ ist, dann gilt: $f \hbox{ injektiv} \iff f \hbox{ surjektiv}$.
  \item [{regulär}] \index{regulär}\index{Abbildung!regulär}
	$\ker(f)=\{0\}$ bzw.~mit der Abbildungsmatrix $A$: $\det(A)\neq0$
  \item [{singulär}] \index{singulär}\index{Abbildung!singulär}
	$\lvert \ker(f) \rvert > 1$ bzw.~mit der Abbildungsmatrix $A$: $\det(A)=0$
  \item [{Injektivität}]
	$f \hbox{ regulär} \iff f \hbox{ injektiv}$
  \item [{Matrixdarstellung}]
	Jede $m\times n$-Matrix liefert eine lineare Abbildung.
  \item [{Isomorphismus}]
	$f$ bijektiv, $A$ quadratisch, $\det(A)\neq0$. Das Bild der Basis von $V$ ist eine Basis von $W$.
  \item [{Urbild}] \index{Abbildung!Urbild}\index{Urbild}
	$f^{-1}:=\{ v \mid f(v)=w\;\forall w\in W\} $. Immer vorhanden.
  \item [{Umkehrabbildung}] \index{Abbildung!Umkehr--}\index{Umkehrabbildung}
	Urbild einer linearen Abbildung: $f(f^{-1}(x))=f^{-1}(f(x))=x$
\end{description}

\subsection{Tensoren}
\index{Tensor}
Ein Tensor beschreibt eine mathematische Funktion, die eine Abbildung von $n$ Vektoren aus einem Vektorraum~$V$ über $\{\mathbb{R}, \mathbb{C}\}$ in den zugrundeliegenden Körper beschreibt. Der Rang bzw.~die Stufe eines Tensors ist die Anzahl der entgegen genommen Vektoren, d.\,h., ein Tensor~$T : V^n \to \mathbb{R}$ hat den Rang~$n$. [Tensoren sind also Operatoren, die mehrere Vektoren verknüpfen.]

Tensoren sind multilinear, d.\,h., sie stellen für jeden entgegen genommene Vektor eine lineare Abbildung dar, also~$T(v_1, \ldots, v_k + cw, \ldots, v_n) = T(v_1, \ldots, v_n) + cT(v_1, \ldots, v_{k-1}, w, v_{k+1}, \ldots, v_n) \forall k$.

\index{k-Form}\index{Determinante}
Die Menge aller multilinearen Abbildungen, die $k$~Vektoren eines Vektorraums über~$K \in \{\mathbb R, \mathbb C\}$ entgegen nehmen, wird als~$\mathcal L^k(K^n, K)$ bezeichnet, die Elemente heißen $k$-Formen. Eine Funktion~$f \in \mathcal L^k(K^n, K)$ heißt antisymmetrisch bzw.~äußere $k$-Form, wenn jede Vertauschung zweier verschiedener Eingangsvektoren von~$f$ eine Negation von~$f$ nach sich zieht. [Tensoren $k$-ter Stufe sind also $k$-Formen, die Determinante ist sogar eine äußere $k$-Form.]

\index{Skalarprodukt}\index{Dyadisches Produkt}
Es lässt sich entsprechend ein Tensor bilden, um das Skalarprodukt~$\langle a, b \rangle$ abzubilden mit $T(a,b) := a\T b$; das Vektorprodukt~$a \times b$ ist jedoch nicht als Tensor darstellbar, da das Ergebnis kein Skalar ist.

\section{Überprüfung auf Linearität}

\index{Linearität}\index{Abbildung!Linearität}
Es gibt zwei Möglichkeiten:
\begin{enumerate}
  \item Obige Definition des Homomorphismus überprüfen.
  \item $f(x)$ als Matrixdarstellung $Ax$, dann prüfen, ob $f(x)=Ax$ gilt (gilt nur für Abbildungen in Vektorräumen über $\mathbb{R}$).
\end{enumerate}

\section{Berechnung der Abbildung}

\index{Abbildung!Berechnung}
Seien $V$, $W$ Vektorräume über einem Körper $K$, $\{v_1,\ldots,v_n\}$ eine Basis von V, sowie $\{w_1,\ldots,w_n\}$ beliebige Vektoren von $W$, dann gilt:
\[ \exists! f:V \to W\;\forall1\leq i\leq n:f(v_{i})=w_{i} \]

Damit ist $f$ durch Abbildung der $v_i$ eindeutig bestimmt.
[$f$ ist also durch die Bilder der Basis von $V$ eindeutig bestimmt.]


\section{Vektorraum aus Matrizen}

Vektorraum der $m\times n$-Matrizen über einem Körper $K$ bzgl.~der Matrixaddition und -Multiplikation mit einem Skalar: $M(m\times n,K)$, alternativ $K^{m\times n}$.
\begin{description}
  \item [{Invertierbarkeit}]
	Wenn $\exists A\in\mathbb{R}^{n\times n}:AA^{-1}=A^{-1}A=E$ gilt, dann heißt $A$ invertierbar
	und $A^{-1}$ Inverse von $A$. $A$ ist genau dann invertierbar, wenn $\det(A)\neq0$.

	Es gilt: $(A^{-1})\t = (A^T)^{-1}$ und $(\lambda A)^{-1}=\frac{1}{\lambda}A^{-1}$.
\end{description}
