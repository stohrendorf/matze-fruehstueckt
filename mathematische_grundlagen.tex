% !TeX root = matze_fruehstueckt.tex
\chapter{\label{chap:mengenlehre}Mengenlehre}

\section{Definitionen}
\index{Cantor}\index{Menge|(}
\noindent
\begin{quotation}
  Eine Menge ist eine Zusammenfassung von bestimmten wohl unterschiedenen Objekten unserer Anschauung oder un\-ser\-es Denk\-ens---welche die Elemente der Menge genannt werden---zu einem Ganzen.
  \begin{flushright} (\noun{Cantor}) \end{flushright}
\end{quotation}

\begin{description}
  \item	[Explizite Aufzählung] \index{Menge!Aufzählung}\index{Aufzählung|see{Menge}}
	$M=\{ a,b,c,d,\ldots\} $
  \item	[Aufzählung nach Eigenschaft] \index{Menge!Eigenschaft}
	$P=\{ x\mid \hbox{$x$ ist Primzahl}\} $
  \item [Leere Menge] \index{Menge!leer}
	$\emptyset$ oder $\{\}$.
	Achtung: $\{\emptyset\}=\{\{\}\}$ ist eine Menge, die eine leere Menge enthält. Es gilt $\forall M:\emptyset\in M$.
  \item [(Echte) Teilmenge] \index{Menge!Teilmenge}
	$A$ ist (echte) Teilmenge von $B$: $A\subset B:=A\neq B\land\forall a\in A:a\in B$.
  \item [Unechte Teilmenge]
	$A$ ist (unechte) Teilmenge von $B$: $A\subseteq B:=\forall a\in A:a\in B$.
    [Da $\subset$ oftmals die Bedeutung von $\subseteq$ hat, kann für echte Teilmengen auch $\subsetneq$ verwendet werden.]
  \item [Durchschnitt (Schnittmenge)] \index{Menge!Durchschnitt}\index{Menge!Schnittmenge}
	$A\cap B:=\{ x\mid x\in A\land x\in B\} $
  \item [Vereinigung] \index{Menge!Vereinigung}
	$A\cup B:=\{ x\mid x\in A\lor x\in B\} $
  \item [Differenzmenge] \index{Menge!Differenzmenge}
	$A\setminus B:=\{ x\mid x\in A\land x\notin B\} $
\end{description}

\begin{figure}[htb]
\caption{\noun{Venn}-Diagramme}

\includegraphics{venn-1.pdf}
\hfill{}
\includegraphics{venn-2.pdf}
\hfill{}
\includegraphics{venn-3.pdf}
\end{figure}

\index{Menge|)}

\section{Rechenregeln}

Prioritäten, absteigend sortiert: Klammern, Komplement, Durchschnitt,
Vereinigung.
\begin{description}
  \item [Kommutativgesetz] \index{Menge!Kommutativgesetz}
	$A\cap B=B\cap A$,
	$A\cup B=B\cup A$
  \item [Assoziativgesetz] \index{Menge!Assoziativgesetz}
	$(A\cap B)\cap C=A\cap(B\cap C)$,
	$(A\cup B)\cup C=A\cup(B\cup C)$
  \item [Distributivgesetz] \index{Menge!Distributivgesetz}
	$A\cap(B\cup C)=(A\cap B)\cup(A\cap C)$,
	$A\cup(B\cap C)=(A\cup B)\cap(A\cup C)$
  \item [Transitivgesetz] \index{Menge!Transitivgesetz}
	$A\subset B\land B\subset C\Rightarrow A\subset C$
  \item [Komplement] \index{Menge!Komplement}
	$\bar{\mathbb{R}}=\emptyset$,
	$\bar{\emptyset}=\mathbb{R}$,
	$\bar{\bar{A}}=A$
  \item [Mächtigkeit/Kardinalität] \index{Menge!Mächtigkeit}\index{Menge!Kardinalität}
	Anzahl der eindeutigen Elemente: $\lvert A \rvert$, manchmal auch $\#A$.
  \item [Potenzmenge] \index{Menge!Potenzmenge}
	Menge aller Teilmengen einer Menge: $\mathcal P(A)$. Es ist $\lvert \mathcal P(A) \rvert = 2 \uparrow \lvert A \rvert$.
  \item [Kreuzprodukt] \index{Menge!Kreuzprodukt}\index{Kreuzprodukt|see{Menge}}\index{Menge!Produktmenge}\index{Menge!kartesisches Produkt}\index{kartesisches Produkt|see{Menge}}
	$A\times B:=\{ (a,b)\mid a\in A\land b\in B\} $
\end{description}

\subsection{\protect\noun{De Morgan}sche Regeln}

\index{de Morgan}
\index{Menge!de Morgan}
\begin{align*}
  A\setminus(B\cap C)  & =(A\setminus B)\cup(A\setminus C)\\
  A\setminus(B\cup C)  & =(A\setminus B)\cap(A\setminus C)\\
  \overline{(A\cup B)} & =\bar{A}\cap\bar{B}\\
  \overline{(A\cap B)} & =\bar{A}\cup\bar{B}
\end{align*}



\section{Relationen}

\index{Relation|see{Menge}}
\index{Menge!Relation}
Eine Relation $R$ auf einer Menge $A$ ist eine (unechte) Teilmenge von $A\times A$.

\begin{description}
  \item [Reflexivität] \index{Reflexivität|see{Äquivalenzrelation}}\index{Menge!reflexiv}
	$\forall a\in A:(a,a)\in R$
  \item [Symmetrie] \index{Symmetrie|see{Menge}}\index{Menge!symmetrisch}
	$\forall a,b\in A:(a,b)\in R \iff (b,a)\in R$
  \item [Transitivität] \index{Transitivität|see{Menge}}\index{Menge!transitiv}
	$\forall a,b,c\in A:(a,b)\in R\land(b,c)\in R\Rightarrow(a,c)\in R$
  \item [Äquivalenzrelation] Eine \index{Äquivalenzrelation|see{Menge}}\index{Menge!Äquivalenzrelation}
	Äquivalenzrelation $\sim$ auf einer Menge $M$ muss reflexiv, symmetrisch und transitiv (\emph{rst}) sein.
	Es gilt dann: $(a,b)\in R \iff a\sim b$.
\end{description}


\subsection{Totale und partielle Ordnung}
\index{totale Ordnung}
\index{Ordnung!total}
Eine totale Ordnung auf einer Menge $M$ ist definiert, wenn $\forall x,y,z \in M$ gilt:
\noindent\begin{center}
\begin{tabular}{rl}
	                                        $x \leq x$ & Reflexivität  \\
	     $x \leq y \land y \leq x\; \Rightarrow\; x=y$ & Antisymmetrie \\
	$x \leq y \land y \leq z\; \Rightarrow\; x \leq z$ & Transitivität \\
	                          $x \leq y \lor y \leq x$ & Totalität
\end{tabular}
\end{center}

\index{partielle Ordnung}
\index{topologische Ordnung}
\index{Ordnung!partiell}
\index{Ordnung!topologisch}
Vernachlässigt man die Forderung nach Totalität (d.\,h., dass nicht jedes Element mit jedem Element vergleichbar ist), so erhält man die Definition einer \emph{partiellen} bzw.~\emph{topologischen} Ordnung.
Z.\,B.~ist das Inhaltsverzeichnis dieses Buches partiell, aber nicht total, nach \enquote{Voraussetzungen} geordnet:
\noindent\begin{quote}
    \noindent Grundlagen $\preceq$ Kurvendiskussion $\preceq$ Vektoranalysis
    \par\hfill $\Rightarrow$ Grundlagen $\preceq$ Vektoranalysis

    \noindent Grundlagen $\preceq$ Matrizen $\preceq$ Lineare Abbildungen
    \par\hfill $\Rightarrow$ Grundlagen $\preceq$ Lineare Abbildungen
    
    \noindent Aber: Matrizen und Vektoranalysis sind nicht vergleichbar.
\end{quote}
[Für partielle Ordnungen werden die Begriffe \enquote{Vorgänger} und \enquote{Nachfolger} statt \enquote{kleiner} bzw.~\enquote{größer} verwendet, sowie gekrümmte Operatoren wie $\prec$, $\preceq$ oder~$\asymp$.]


\section{Gruppe}

Eine nicht-leere Menge $M$, auf der eine Verknüpfung~$\circ$ definiert
ist, heißt \index{Gruppe}Gruppe, wenn für $\forall x,y,z\in M$
gilt:
\begin{description}
  \item [Abgeschlossenheit] \index{Abgeschlossenheit|see{Gruppe}}\index{Gruppe!abgeschlossen}
	$x\circ y\in M$
  \item [Assoziativität] \index{Assoziativität|see{Gruppe}}\index{Gruppe!assoziativ}
	$(x\circ y)\circ z=x\circ(y\circ z)$
  \item [Neutralelement] \index{Neutralelement|see{Gruppe}}\index{Gruppe!Neutralelement}
	Es existiert \emph{genau ein} Element $n\in M$, für das gilt: $x\circ n=n\circ x=x$
  \item [Inverses Element] \index{inverses Element|see{Gruppe}}\index{Gruppe!inverses Element}
	Zu jedem $x\in M$ existiert ein $x^{-1}\in M$, für das
	$x\circ x^{-1}=x^{-1}\circ x=n$ gilt.
	[Das inverse Element $x^{-1}$ ist nicht mit dem Kehrwert zu verwechseln.]
\end{description}

Mit dieser zusätzlichen Bedingung heißt die Gruppe kommutativ oder \noun{Abel}sch:
\begin{description}
  \item [Kommutativ] \index{Kommutativ|see{Gruppe}}\index{Abelsch|see{Gruppe}}\index{Gruppe!kommutativ}\index{Gruppe!Abelsch}
	$x\circ y=y\circ x$
\end{description}

\section{Ring}

Eine nicht-leere Menge $M$ heißt \index{Ring}Ring, wenn auf ihr zwei Verknüpfungen $\oplus$ und $\odot$ mit folgenden Eigenschaften definiert sind:

\begin{enumerate}
  \item $M$ ist eine kommutative Gruppe bezüglich $\oplus$.
  \item Ohne das Neutralelement bzgl.~$\oplus$ erfüllt der Rest von $M$ alle Axiome einer Gruppe bezüglich $\odot$, jedoch \emph{nicht} das Axiom des inversen Elements.
  \item Es gelten die Distributivgesetze ($\forall x,y,z\in M$):
	\begin{align*}
	  (x\oplus y)\odot z & =(x\odot z)\oplus(y\odot z)\\
	  x\odot(y\oplus z) & =(x\odot y)\oplus(x\odot z)
	\end{align*}
\end{enumerate}

[Das neutrale Element bezüglich der \enquote{Addition} $\oplus$ wird oft mit~0 bezeichnet, das neutrale Element bezüglich der \enquote{Multiplikation} $\odot$ oft als~1.]


\section{Körper}

Eine nicht-leere Menge $M$ heißt \index{Körper}Körper, wenn sie die Eigenschaften des Ringes erfüllt und zusätzlich ein inverses Element bezüglich $\odot$ besitzt.


\section{Abbildungen}

\index{Abbildung|(}
\begin{description}
  \item [Funktion/Abbildung] \index{Funktion}
	$f:A\rightarrow B$ oder $f:A\ni a\rightarrow f(a)\in B$
  \item [Definitionsbereich] \index{Definitionsbereich}
	Das $A$ bei $f:A\rightarrow B$.
  \item [Zielmenge] \index{Zielmenge}
	Das $B$ bei $f:A\rightarrow B$.
  \item [Bildmenge] \index{Bildmenge}
	$f(A)\subseteq B$, auch: Wertemenge\slash{}Bild von $A$ bzgl.~$f$.
  \item [Urbild] \index{Urbild}
	$f^{-1}(b):=\{a\in A\mid f(a)=b\}\subseteq A$, Urbild von $b\in B$.
  \item [Injektivität] \index{Abbildung!injektiv}\index{Injektivität}
	$f(u)=f(v) \implies u=v$.
  \item [Surjektivität] \index{Abbildung!surjektiv}\index{Surjektivität}
	$f(A)=B$.
  \item [Bijektivität] \index{Abbildung!bijektiv}\index{Bijektivität}
	Injektiv und surjektiv.
\end{description}

\subsection{Homomorphismus}

\index{Abbildung!linear}\index{Abbildung!Homomorphismus}\index{Homomorphismus}
Seien $A$ und $B$ Vektorräume über einem Körper $K$\@.
Eine Abbildung $f : A\rightarrow B$ heißt \emph{Homomorphismus} bzw.~\emph{lineare Abbildung}, falls gilt:
\begin{description}
  \item [Additivität] $\forall x,y\in A : f(x+y)=f(x)+f(y)$
  \item [Homogenität] $\forall x\in A\;\forall\lambda\in K : f(\lambda\cdot x)=\lambda\cdot f(x)$
\end{description}

Die Menge der linearen Abbildungen von $A$ nach $B$ wird mit $\hom(A,B)$ bezeichnet.
\begin{description}
  \item [Kern] \index{Abbildung!Nullraum}\index{Nullraum}\index{Abbildung!Kern}\index{Kern}
	von $f$ bzw.~\emph{Nullraum} von $f$ ist definiert als $\ker(f) := \{ x \in A \mid f(x)=0 \}$.
	Enthält entweder genau ein Element (die 0) oder unendlich viele.
  \item [Regulär] \index{regulär}\index{Abbildung!regulär}
	$\ker(f)=\{0\}$
  \item [Singulär] \index{singulär}\index{Abbildung!singulär}
	$\lvert\ker(f)\rvert = \infty$
  \item [Isomorphismus] \index{Isomorphismus}\index{Abbildung!isomorph}
	ist ein bijektiver Homomorphismus.
\end{description}

Siehe dazu auch \cref{chap:lineare-abbildungen}.

\index{Abbildung|)}


\section{Modulo-Arithmetik}

\index{Modulo-Arithmetik}\index{Restklasse}\index{Kongruenz|see{Restklasse}}
Zwei Zahlen $a,b\in\mathbb{Z}$ heißen zu einer festen Zahl $m\in\mathbb{N}\setminus\{0\}$ kongruent modulo $m$, wenn gilt:
\[ a\equiv b \pmod m \quad \iff \quad \lvert a-b \rvert = n\cdot m,\,n\in\mathbb{N} \]

Das heißt, dass die Differenz beider Zahlen $a$ und $b$ durch $m$ teilbar ist.

Weiter gelten:
\begin{align*}
  (x +     y) \bmod p & =((x\bmod p)+(y\bmod p)) \bmod p \\
  (x \cdot y) \bmod p & =((x\bmod p)\cdot(y\bmod p))\bmod p\\
  x^n \bmod p         & = (x\bmod p)^n \bmod p
\end{align*}



\chapter{\label{chap:Kombinatorik}Kombinatorik}

\index{Kombinatorik|(}
Die Kombinatorik beschreibt eine Auswahl von $k$ Objekten aus $n$ Elementen, welche in einer Ergebnismenge\index{Ergebnismenge} zusammengefasst werden.
Gesucht ist dann die Mächtigkeit\index{Mächtigkeit} $a$ der Ergebnismenge.

\begin{description}
  \item [Permutation] \index{Permutation}
	Berücksichtigung der Reihenfolge
  \item [Kombination] \index{Kombination}
	Ignorieren der Reihenfolge
\end{description}

\noindent\begin{center}
\begin{tabular}{rcc}
              & \bfseries mit Wdh.
              & \bfseries ohne Wdh. \\
  \bfseries Perm.
              & $\left|\mathrm{PER}^{\mathrm{MW}}(n,k)\right| = n^k$
              & $\left|\mathrm{PER}^{\mathrm{OW}}(n,k)\right| = \vecc(n;k) k!$ \\[1.6667em]
  \bfseries Komb.
              & $\left|\mathrm{KOM}^{\mathrm{MW}}(n,k)\right| = \vecc(n+k-1;k)$
              & $\left|\mathrm{KOM}^{\mathrm{OW}}(n,k)\right| = \vecc(n;k)$
\end{tabular}

\vspace{0.66667em}
\emph{Betragsstriche beachten! Die Funktionen geben die Kombinationen bzw.~Permutationen zurück.}
\end{center}

\begin{description}
  \item [Teilchen-/Fächermodell] \index{Teilchenmodell}\index{Fächermodell}\index{Grundmenge}
	Grundmenge: Fächer, Auswahl: Besetzung der Fäch\-er mit Teilchen.

	Mächtigkeit der Grundmenge ist Anzahl der Fächer, Experiment erfolgt durch Verteilen der Teilchen in die Fächer.

  \item [Urnenmodell] \index{Urnenmodell}
	Grundmenge: Kugeln in Urne, Auswahl: Ziehung der Kugeln.

	Mächtigkeit der Grundmenge ist Anzahl der Kugeln, Experiment besteht aus Ziehung der Kugeln.
\end{description}

\index{Kombinatorik|)}


\chapter{Komplexe Zahlen}
\begin{description}
  \item [Definition] \index{Komplexe Zahl}
	$z = x+iy = r\cdot e^{i\varphi} = r\cdot (\cos(\varphi)+i\sin(\varphi))$
  \item [Realteil] $\Re(z) = \mathrm{Re}(z) = x$
  \item [Imaginärteil] $\Im(z) = \mathrm{Im}(z) = y$
  \item [Konjugiert komplex] \index{Komplexe Zahl!konjugiert komplex}\index{konjugiert komplex|see{Komplexe Zahl}}
	$\bar{z} = x-iy$
  \item [Betrag] \index{Komplexe Zahl!Betrag}\index{Betrag|see{Komplexe Zahl}}
	$\lvert z \rvert = \sqrt{x^2+y^2} = \sqrt{z\cdot\bar{z}}$
  \item [Potenzen] \index{Komplexe Zahl!Potenzen}
	$z^n = r^n \cdot \exp(n\cdot i\varphi)$
  \item [Wurzeln] \index{Komplexe Zahl!Wurzeln}
	$\sqrt[n]{z} = \sqrt[n]{r} \cdot \exp \bigl( (i\varphi+2k\pi i)/n\bigr), \quad 0 \leq k < n$
\end{description}

\section{Polarkoordinaten}

\index{Komplexe Zahl!Polarkoordinaten}\index{Polarkoordinaten|see{Komplexe Zahl}}
\[
  \varphi=\begin{cases}
     \arccos(\mathrm{Re}(c) / \lvert c \rvert) & \hbox{für $\mathrm{Im}(c) \geq 0$}\\
    -\arccos(\mathrm{Re}(c) / \lvert c \rvert) & \hbox{für $\mathrm{Im}(c) < 0$}
  \end{cases}
\]


\chapter{Sonstiges}

\section{\label{sec:intervalle}Intervalle}

\begin{align*}
  \left[a;b\right]   & =\{ x\mid a\leq x\leq b\} =a\leq x\leq b\\
  \left[a;b\right[   & =\{ x\mid a\leq x<b\}     =a\leq x<b\\
  \left]a;b\right]   & =\{ x\mid a<x\leq b\}     =a<x\leq b\\
  \left]a;b\right[   & =\{ x\mid a<x<b\}         =a<x<b
\end{align*}
[Für $a=-\infty$ bzw.~$b=\infty$ sind jeweils offene Grenzen zu benutzen.]


\section{\label{hornerschema}\noun{Horner}schema}
\index{Hornerschema}
Das \noun{Horner}schma kann zur Abspaltung von Linearfaktoren in Polynomen verwendet werden.
Sei $f(x)=\sum_{0 \leq i \leq n} a_i x^i$ ein Polynom $n$-ten Grades, und $x-x_0$ ein abzuspaltender Linearfaktor, so lässt sich $f(x)$ umschreiben als $f(x)=r+(x-x_0)\cdot\sum_{0 \leq i < n} b_i x^i$:

\noindent\begin{center}
\begin{tabular}{l|l|l|l|l}
	$a_n$         & $a_{n-1}$              & $a_{n-2}$              & $\cdots$ & $a_0$                \\ \hline
	$x_0=\ldots$  & $+(b_{n-1} \cdot x_0)$ & $+(b_{n-2} \cdot x_0)$ & $\cdots$ & $+(b_{0} \cdot x_0)$ \\
	$b_{n-1}=a_n$ & $=b_{n-2}$             & $=b_{n-3}$             & $\cdots$ & $=r$
\end{tabular}
\end{center}

\emph{Achtung:} Fehlende $a_i$ sind mit~$0$ in die Tabelle einzutragen!

[Wenn $r=0$ ist, ist $x_0$ eine Nullstelle von $f(x)$.]

Beispiel für $x_0=2$ und $f(x)=x^5 - 4x^4 + 4x^3 + 3x^2 - 8x + 4$:
\noindent\begin{center}
\begin{tabular}{r|r|r|r|r|r}
	$1$     & $-4$ & $4$  & $3$ & $-8$ & $4$  \\ \hline
	$x_0=2$ & $2$  & $-4$ & $0$ & $6$  & $-4$ \\
	$1$     & $-2$ & $0$  & $3$ & $-2$ & $0$
\end{tabular}
\end{center}
Damit: $f(x) = (x^4 - 2x^3 + 0x^2 + 3x - 2)\cdot(x-2) + 0$, also ist $2$ eine Nullstelle von $f(x)$.


\section{Regel von \noun{L'Hospital}}
\index{L'Hospital}
\index{Regel!von L'Hospital}
Wenn eine Funktion $u(x)=f(x)/g(x)$ als Ergebnis $0/0$ oder $\pm \infty/\infty$ liefert, ist $u(x)=f'(x)/g'(x)$.
[\emph{Achtung:} Die Fälle $0/\infty$ und $\infty/0$ sind hiervon \emph{nicht} betroffen.]



\chapter[Vollständige Induktion]{Beweis durch vollständige Induktion}

\CheckedBox{} Beispiel vorhanden auf \cpageref{sec:bsp-Vollständige-Induktion}.
\begin{description}
  \item [Induktionsanfang (IA)] \index{Induktion!vollständige --}
	Beweise, dass eine Aussage $f(n)=g(n)$ für ein ge\-wählt\-es $n_{0}$ gilt (i.\,d.\,R.~für $n_{0}=1$).
  \item [Induktionsvoraussetzung (IV)]
	Gleichsetzen der rekursiven Formel und der expliziten Formel, also $f(n)=g(n)$ explizit aufschreiben.
  \item [Induktionsbehauptung (IB)]
	Satz: \enquote{Wenn $f(n)=g(n)$ für ein beliebiges aber festes $n$ gilt, so gilt die Aussage auch für $n+1$.}
  \item [Induktionsschluss (IS)]
	Überprüfen, ob $f(n+1)=g(n+1)$ unter Verwendung der IV stets eine wahre Aussage ergibt, d.\,h., eine Seite der Aussage muss mit Hilfe der IV in die andere Seite umgeformt werden.
	Beim Einsetzen der IV sollte über dem Gleichheitszeichen \enquote{IV} stehen.
	Schlusssatz: \enquote{Damit ist die IV bewiesen.}
\end{description}


\chapter{\noun{Bool}sche Algebra und Aussagenlogik}

Beachte Ähnlichkeiten zur Mengenlehre, \cpageref{chap:mengenlehre}.

\begin{description}
    \item[Aussage] $A$, $B$, \ldots bzw.~$a(x)$, $b(x)$\ldots
    \item[Tautologie] $\models A$ heißt, dass $A$ unabhängig seiner abhängigen Variablen wahr ist.
    \item[Negation] $\neg A$ oder $\overline{A}$
    \item[Bikonditional] $A \leftrightarrow B$; $A$ genau dann, wenn $B$. Falls $A \leftrightarrow B$ eine Tautologie ist,
                         schreibt man $A \Leftrightarrow B$.
    \item[Konjunktion] $A \land B$; $\neg(A \land B) \leftrightarrow (\neg A \lor \neg B)$
    \item[Disjunktion] $A \lor B$; $\neg(A \lor B) \leftrightarrow (\neg A \land \neg B)$
    \item[Implikation] $(A \rightarrow B) := (\neg A \lor B)$, $A \vdash B$, $B \subset A$, auch
                       \enquote{Konditional} oder \enquote{Subjunktion} genannt; $A$ ist eine hinreichende
                       Bedingung für $B$, $B$ ist eine notwendige Bedingung für $A$. Falls $A \rightarrow B$ eine Tautologie
                       ist, schreibt man $A \Rightarrow B$ oder $A \models B$.
    \item[Kontravalenz] $\neg(A \leftrightarrow B) \leftrightarrow (A \oplus B)$, ausschließendes Oder.
\end{description}

[\enquote{Doppelstrich-Operatoren} wie $\models$, $\Rightarrow$ oder $\Leftrightarrow$ drücken damit aus, dass eine gegebene
Aussage unabhängig ihrer abhängigen Variablen wahr ist; die analogen Operatoren $\vdash$, $\rightarrow$ und $\leftrightarrow$ sagen lediglich aus, dass die entsprechende Aussage syntaktisch hergeleitet werden kann, d.\,h.~$A \rightarrow \neg A$ ist korrekt, jedoch $A \Rightarrow \neg A$ nicht.]


\begin{table}[htb]
    \centering\begin{tabular}{rcl}
    	             $(A \lor A)$ &   $\rightarrow$   & $A$                            \\
    	                      $B$ &   $\rightarrow$   & $(A \lor B)$                   \\
    	             $(A \lor B)$ &   $\rightarrow$   & $(B \lor A)$                   \\
    	    $(A \lor (B \lor C))$ &   $\rightarrow$   & $(B \lor (A \lor C))$          \\
    	      $(A \rightarrow B)$ &   $\rightarrow$   & $((C \lor A) \lor (C \lor B))$ \\
    	$(\neg(A \rightarrow B))$ & $\leftrightarrow$ & $(\neg B \rightarrow \neg A)$
    \end{tabular}
    \caption{Axiome der Aussagenlogik}
\end{table}
