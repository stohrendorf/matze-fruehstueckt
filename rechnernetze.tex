% !TeX root = matze_fruehstueckt.tex
\Todo{Stichworte hinzuf{\"u}gen}

\Todo{Netwerktypen}

\Todo{WLAN}

\Todo{Datex, ISDN, DSL}


\chapter{Netzwerktypen und -Begriffe}

\begin{table}[h]
\centering
\begin{tabular}{rl}
\bfseries Größe (max.) & \bfseries Name \\
1 m & PAN---Personal Area Network \\
100 m & LAN---Local Area Network \\
10 km & MAN---Metropolitan Area Network \\
1000 km & WAN---Wide Area Network \\
10\,000 km & Internet
\end{tabular}
\caption{Netzwerkgrößen}
\end{table}

\begin{description}
\item [{Switch}] Netzwerkverteiler, welcher Datenpakete gezielt routen kann.
\item [{Hub}] Netzwerkverteiler, welcher Datenpakete an alle an\-ge\-schloss\-en\-en Geräte verteilt (wie ein Broadcast).
\item [{Router}] Netzwerkverteiler, welcher Informationen über Netzwerktopologien kennt und Datenpakete auch an nicht direkt angeschlossene Geräte ,,routen`` kann.
\item [{Backbone}] Router bzw.~Switch, der zwischen großen Teilnetzen vermittelt.
\end{description}

\section{ISO/OSI Referenzmodell}
\begin{enumerate}
\item Physische Schicht (Bitübertragung; Spannungsregelung/Bitcodierung in Kabeln, Pinbelegungen, Übertragungsrichtung, etc.)
\item Sicherungsschicht (Datenlink; Flusskontrolle, Prüfsummenbildung und Sich\-er\-ung der fehlerfreien Übertragung, konfliktfreier Zugriff auf Leitung, etc.)
\item Vermittlungsschicht (Netzwerk; Findung von Routing-Informationen, Multiplexing logischer Verbindungen auf einer physischen Leitung, etc.)
\item Transportschicht (Segmentierung von Datenströmen, Verbindungsaufbau, Aushandlung von Übertragungsgeschwindigkeit, etc.)
\item Sitzungsschicht (Dialogkontrolle, z.\,B.~Übermittlung des Wiederaufnahmepunktes einer abgebrochenen Dateiübertragung)
\item Darstellungsschicht (Art der Datenübertragung: XML, JSON, ASCII oder Unicode, etc.)
\item Anwendungsschicht (Übertragungsprotokoll: HTTP, FTP, SSH, etc.)
\end{enumerate}

\subsection{Praxis}

Die Schichten 1.~und 2.~sind im Netzwerktreiber implementiert, die Schichten 5.~bis 7.~in der Anwendung.
Die Schichten 3.~und 4.~sind als IP (\emph{Internet Protocol}) und TCP (\emph{Transmission Control Protocol}, verbindungsorientiert) bzw.~UDP (\emph{User Datagram Protocol}, verbindungslos) implementiert.
\begin{description}
\item [{TCP}] Segmentierter Datenstrom, Flusskontrolle, Daten werden garantiert komplett übertragen und in der richtigen Reihenfolge wieder zusammen gesetzt.
\item [{UDP}] Segmentierter Datenstrom, Pakete können allerdings in unterschiedlicher Reihenfolge oder gar nicht übertragen werden.
\item [{Socket}] Direkter Zugriff auf die Transportschicht; IP, Portnummer und Verbindungsart (TCP/UDP) müssen angegeben werden, das Protokoll der Übertragung definiert die Anwendung.
\item [{XML}] Extensible Markup Language; menschenlesbar, unabhängig von Little/Big Endian.
\item [{Verbindungslos}] Pakete werden einfach ins Netz ,,geschossen``, ohne zu überprüfen, ob die Pakete beim Empfänger ankommen.
\item [{Verbindungsorientiert}] Verbindungen werden aufgebaut, danach werden Pakete übermittelt und auf die Bestätigung des Empfängers gewartet, danach wird die Verbindung wieder abgebaut.
\end{description}

\chapter{Datenübertragung}
\begin{description}
\item [{Virtuelle~Leitung}] Mehrere Verbindungen auf einer physischen
Leitung.

\begin{itemize}
\item Aushandlung von Übertragungsraten, Puffergrößen und Route.
\item Route nach Aushandlung fest, Paketroute durch Kennung im Paketheader identifizierbar.
\end{itemize}
\item [{Frequency~Division~Multiplexing}] mehrere Datenströme werden auf verschiedene Frequenzen aufmoduliert.
\item [{Time~Division~Multiplexing}] mehrere Datenströme werden abwechselnd übertragen (vgl.~Multithreading).
\item [{Statistical~Multiplexing}] Aufteilung der Ressourcen je nach Bedarf.
\item [{Ausfallsicherheit}] Pakete werden im Internet über redundante Wege übermittelt, daher ist die Reihenfolge beim Empfänger nicht garantiert.
\end{description}

\section{TCP}

Verbindungsorientiert, Adressierung über Ports, reihenfolgetreu, fehlerfrei.
\begin{itemize}
\item Segmentierung in Pakete von max.~64\,kB (Datensegmente, Kontrollsegmente)
\item Bytestrom
\item Fehlerbehandlung
\item Flusskontrolle
\item Voll-Duplex
\item Punkt-zu-Punkt
\end{itemize}

\subsection{Ports}

\begin{table}[h]
\centering
\begin{tabular}{cc}
\bfseries Bereich & \bfseries Bedeutung \\
0--1\,023 & Reserviert (HTTP, FTP,\ldots) \\
1\,024--49\,151 & Quasi-reservierte Ports (de-facto-Standards) \\
49\,152--65\,535 & Benutzerdefinierbar
\end{tabular}

\caption{TCP Ports}
\end{table}


\subsection{Three-Way-Handshake}
\begin{enumerate}
\item Verbindungsanfrage; \code{CR(seq=x)}
\item Server bestätigt Verbindungsaufbau; \code{ACK(seq=y; ack=x+1)}
\item Client bestätigt Verbindungsaufbau; \code{ACK(seq=x+1, ack=y+1)}
\item Danach Beginn der Datenübertragung mit fortlaufenden Sequenznummern und Quittungsübermittlung zwecks Fehlervermeidung.

\begin{itemize}
\item Aufteilung des Datenstroms in Segmente.
\item Segmente enthalten Sequenznummern mit Position der Daten im gesamten Datenstrom.
\item Für jedes erfolgreich übermittelte Segment wird ein \code{ACK} zurückgesendet.
\item Letztes Segment enthält das \code{FIN-Flag}.
\end{itemize}
\end{enumerate}

\subsection{Flusskontrolle}
\begin{itemize}
\item Vereinbarung eines Übertragungsfensters (Größe des Puffers auf der Empfangsseite).
\item Der Sender sendet maximal so viel wie in den Puffer des Empfängers passt und wartet dann auf die Quittungen; nach jeder Quittung wird das nächste Paket versendet.
\end{itemize}

\subsection{TCP Header}

\begin{table}[h]
\centering
\begin{tabular}{|c|c|c|c|c|}
	\hline
	\multicolumn{2}{|c|}{0} & 8 & 16 & 24\tabularnewline
\hline 
\hline 
\multicolumn{3}{|c|}{Source Port} & \multicolumn{2}{c|}{Destination Port}\tabularnewline
\hline 
\multicolumn{5}{|c|}{Sequence Number}\tabularnewline
\hline 
\multicolumn{5}{|c|}{Acknowledgement Number}\tabularnewline
\hline 
HL & Res. & 6 Flags & \multicolumn{2}{c|}{Window Size}\tabularnewline
\hline 
\multicolumn{3}{|c|}{Checksum} & \multicolumn{2}{c|}{Urgent Pointer}\tabularnewline
\hline 
\multicolumn{4}{|c|}{Options} & Padding\tabularnewline
\hline 
\multicolumn{5}{|c|}{Data\ldots}\tabularnewline
\hline
\end{tabular}
\caption{TCP Header}
\end{table}

\begin{itemize}
	\item Header ist 20 Bytes groß, zzgl.~Optionen und Padding.
	\item Sequenz-/Quittungsnummer zählen einzelne Bytes; Quittungsnummer gibt das nächste erwartete Byte an.
	\item HL: Header Length; Anzahl der im Header übertragenen Bytes geteilt durch 4.
	\item Res.: reservierte Flags.
	\item Flags:
	\begin{itemize}
		\item \code{URG}: Urgent; wichtige Daten
		\item \code{ACK}: Quittung
		\item \code{PSH}: Push; direkte Weiterleitung
		\item \code{RST}: Reset; Zurücksetzen der Verbindung
		\item \code{SYN}: Verbindungsaufbau
		\item \code{FIN}: Verbindungsabbau
	\end{itemize}
	\item Window Size: Größe des Empfangspuffers in Bytes
	\item Checksum: Prüfsumme der Daten
	\item Urgent Pointer: Daten ohne Beachtung der Flusskontrolle senden; Wert ist die absolute Byteposition im gesamten Datenstrom
\end{itemize}
Weitere Informationen, die beim Verbindungsaufbau ausgehandelt werden:
\begin{itemize}
	\item MMS (Maximum Segment Size): Maximale Segmentgröße
	\item Window Scale: Skalierungsfaktor für die Fenstergröße, maximal 14, Faktor errechnet sich als $2^n$.
	\item Fehlerbehandlungsmechanismus
	\item Slow Start/Congestion Avoidance
	\begin{itemize}
		\item Exponentielle Steigerung der Anzahl übertragener Segmente bis die Window Size erreicht ist.
		\item Danach nur noch lineares Wachstum.
		\item Reduktion der Anzahl bei Paketverlusten.
	\end{itemize}
\end{itemize}

\section{UDP}

Verbindungslos, Adressierung über Ports, nicht reihenfolgetreu, nicht fehlerfrei.
\begin{itemize}
\item Nur Ports, Message Length und Checksum im Header, damit nur 8 Byte Header.
\item Keine Kontrollmechanismen: Reihenfolge der Pakete sowie Paketverlust und -duplizierung erst auf Anwendungsebene abfangbar.
\end{itemize}

\chapter{Internet}


\section{IP}
\begin{description}
\item [{Class~A}] Erstes Byte für Netzmaske, fängt mit dem Bitwert 0 an.
\item [{Class~B}] Erste 2 Bytes für Netzmaske, fängt mit den Bitwerten 10 an.
\item [{Class~C}] Erste 3 Bytes für Netzmaske, fängt mit den Bitwerten 110 an.
\item [{Class~D}] Multicast, fängt mit den Bitwerten 1110 an.
\item [{Class~E}] Reserviert, fängt mit den Bitwerten 1111\,0 an.
\end{description}

\subsection{Routing}

Router enthält Informationen, über welchen Netzwerkanschluss welche IPs erreicht werden können.

Spezielle Adressen:
\begin{description}
\item [{0.0.0.0}] Eigener Host.
\item [{0\ldots{}HOST}] Host im eigenen Netz.
\item [{255.255.255.255}] Broadcast (lokal)
\item [{NETMASK.111\ldots{}111}] Broadcast (entfernt)
\item [{127.x.x.x}] Loopback
\end{description}

