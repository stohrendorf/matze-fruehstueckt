% !TeX root = matze_fruehstueckt.tex
\newminted{sql}{linenos,numbersep=5pt,frame=lines,framesep=2mm}
\newmintinline[sqli]{sql}{}
%\newcommand{\sqli}{\mintinline{sql}}

\Todo{Recherchieren und überarbeiten}


\chapter{Dateibasierte Datenverwaltung}


\section{Definitionen}
\begin{description}
  \item [{Daten}] \index{Daten}\index{Syntax}
	Strukturierte Objekte (Syntax), die durch Anwendung auf Rechner verarbeitet werden
  \item [{Information}] \index{Shannon}\index{Semantik}\index{Information}
	Daten mit Bedeutung (Semantik) für Empfänger

	\noun{Shannon}: Wert---Info: Zuwachs der nach Kenntnis der Information beantwortbaren Ja\slash{}Nein-Fragen

  \item [{Wissen}] \index{Wissen}
	Verknüpfung von Informationen, um Prognosen/Erklärungen abzugeben
\end{description}

\subsection{Informationen und Daten}
\begin{itemize}
  \item Daten sind der Rohstoff, aus dem Informationen abgeleitet werden
  \item Informationen basieren auf Daten
  \item Gleiche Daten sind stets verschiedene Informationen---??? Backspace{} ,,stets`` konnte nicht genau aus den Aufzeichnungen entnommen werden!\Todo{Fix me}
\end{itemize}

\subsection{Festplatte}
\begin{itemize}
  \item \index{Achse}\index{Festplatte}
	$x$ Platten rotieren um eine Achse
  \item \index{Lesekopf}\index{Schreibkopf}
	Schreib-/Leseköpfe schweben (ca.~1\,\textmu{}m Abstand)
  \item \index{Positionierungslogik}
	Elektronischer Bereich: Positionierungslogik, Schreib-/Leseverstärker, De-/Kodierungslogik, Ein-/Ausgabelogik
\end{itemize}
\begin{description}
  \item [{Platte}] \index{Oberfläche}
	Oberfläche
  \item [{Spur}] \index{Spur}\index{Sektor}
	Sektor
  \item [{Spuren}]
	Konzentrische Kreise mit gleichem Abstand zur Aufnahme von Informationen
  \item [{Zylinder}] \index{Plattenstapel}\index{Zylinder}
	Übereinander liegende Spuren des Plattenstapels, die mit einer Bewegung der Schreib-/Leseköpfe erreicht werden können.
	Alle Spuren im Zylinder können ohne neue Positionierung verarbeitet werden.
  \item [{Sektoren}]
	Kleinste adressierbare Einheit einer Festplatte. Jede auf einer Festplatte gespeicherte Datei belegt eine ganzzahlige Anzahl von Sektoren.
\end{description}

\subsubsection{Sektorzugriffe}
\begin{description}
  \item [{Mittlere~Suchzeit}] \index{Mittlere!Suchzeit}\index{Suchzeit}
	Zeit, um $\nicefrac{1}{3}$ aller Zylinder zu überfahren
  \item [{Mittlere~Drehwartezeit}] \index{Mittlere!Drehwartezeit}\index{Drehwartezeit}
	Zeit, die eine halbe Umdrehung benötigt
  \item [{Mittlere~Zugriffszeit}] \index{Mittlere!Zugriffszeit}\index{Zugriffszeit}
	Summe aus mittlerer Suchzeit und Drehwartezeit
\end{description}

\subsection{Dateisysteme}

\index{Dateisystem}\index{Betriebssystem}
Betriebssystem unterstützen den Benutzer durch Dateisysteme.\footnote{
	Siehe auch \vref{sec:Dateisysteme}.
}
\begin{description}
  \item [{Dateien}] \index{Datei}
	Ansammlung von logisch zusammenhängenden Daten
  \item [{Dateisystem}] \index{Datei!--system}
	Liefert die Struktur zur Speicherung von Dateien auf Datenträger
  \item [{Dateidaten}] \index{Datei!--daten}
	müssen nicht physisch zusammenhängend gespeichert werden
  \item [{Dateieigenschaften~und~-referenzen}] \index{Metadaten}
	werden in einer Verwaltungsstruktur (\emph{Metadaten}) gespeichert
\end{description}

\subsubsection{Unix/Windows}
\begin{description}
  \item [{Daten}]     \index{Daten}\index{Linux}\index{Windows}\index{Unix}Unstrukturierte Folge von Bytes
  \item [{Bedeutung}] Anwendungsspezifisch
\end{description}

Die Anwendungen nutzen die Abstraktion des Betriebssystems zur Abspeicherung
der Informationen als Daten. Programmiersprachen bieten i.\,d.\,R.~entsprechende
Abstraktionen:
\index{Java}\index{Stream}
\begin{itemize}
  \item Ein-/Ausgabe: Stream
  \item Streams entkoppeln die Anwendung von den Betriebssystem-Details
  \item Streams kapseln den Dateizugriff in Methoden (lesen, schreiben)
  \item In Java: \code{InputStream} und \code{OutputStream} zur Kapselung
\end{itemize}
\begin{description}
  \item [{Probleme}] \index{Daten!Inkonsistenz}Zeit-/Ressourcenintensiv, Zugriffe mehrerer Anwendungen führen zu Dateninkonsistenz
\end{description}

\subsection{Vermeidung von Redundanz}
\begin{description}
  \item [{Ziel}] \index{Syntax}\index{Abhängigkeit}\index{Redundanz}
	Reduktion der Abhängigkeiten zwischen Anwendung von Syntax der Daten (bei Erhalt der ableitbaren Informationen).
\end{description}
Auftretende Probleme:
\begin{itemize}
  \item Änderung der logischen/physischen Struktur der Daten erfordert Änderung der Anwendungen.
  \item \index{Zugriffskontrolle}\index{Daten!--sicherheit}\index{Daten!--schutz}
	Datenschutz und -sicherheit: unflexible Zugriffskontrolle, Systemabstürze, Hardwaredefekte
  \item Unflexible Gewinnung von Informationen
  \item Verarbeitung großer Datenmengen
\end{itemize}

\subsection{Datenunabhängigkeit}
\begin{description}
  \item [{Ziel}] \index{Daten!Unabhängigkeit}
	Strukturelle Änderungen dürfen sich nicht unmittelbar auf Anwendungsprogramme auswirken.
  \item [{Logisch}]
	Daten als eigenständige Ressource modellieren (abstrakte Syntax, z.\,B.~SQL); Nutzer sieht nur, was er sehen muss\slash{}möchte
  \item [{Physisch}]
	Flexibles Abspeichern der Daten
\end{description}

\chapter{Aufgaben und Aufbau eines Datenbank-Systems}


\section{Konzept der Datenbank}
\begin{itemize}
  \item Relevante Daten werden nicht mehr indirekt durch die Anwendungsentwickler modelliert, sondern eigenständig und anwendungsübergreifend
  \item \index{Daten!--struktur}\index{Ressource}Daten als eigene Ressource (analog zur Datenstruktur einer Anwendung)
  \item \index{Softwareschicht}Eine eigene Softwareschicht stellt der Anwendung die benötigten Daten zur Verfügung.
\end{itemize}

\section{Definitionen}
\nomenclature{DBS}{Database System; Kombination von Datenbank und DBMS}
\nomenclature{DBMS}{Database Management System; Gesamtheit aller Programme zur Erzeugung, Verwaltung und Manipulation einer Datenbank}
\begin{description}
  \item [{Datenbank}] \index{Redundanz}\index{Metadaten}\index{Datenbestand}\index{Datenbank}
	Integrierter, persistenter Datenbestand, einschließlich aller Metadaten, die einer Gruppe von Nutzern zur Verfügung
	gestellt wird und durch spezielle Software möglichst redundanzfrei verwaltet wird.
  \item [{Database~Management~System (DBMS)}] \index{DBMS}
	Gesamtheit aller Programme zur Erzeugung, Verwaltung und Manipulation einer Datenbank
  \item [{Datenbanksystem~(DBS)}] \index{DBS}
	Kombination von Datenbank und DBMS
\end{description}

\section{Trennung von Verwaltung und Verwendung}

Abstraktionsebenen:
\begin{itemize}
  \item Wie ist die Struktur für die einzelne Anwendung?
  \item Wie sind die Daten insgesamt strukturiert?
  \item Wie werden die Daten effizient gespeichert?
\end{itemize}

\subsection{Relationales Modell}
\begin{itemize}
  \item Daten werden in Form von \emph{Relationen} (Tabellen) gespeichert
      \begin{itemize}
	\item Zeile: \emph{Tupel}
	\item Spalten: \emph{Attribute}
	\item Eine Zeile wird über gewählte Attributkombination eindeutig identifiziert (\emph{Primärschlüssel})
      \end{itemize}
  \item Attribute sind atomare Datentypen
  \item Sachlogische Zusammenhänge: Verknüpfung der Tabellen, redundante Speicherung der Primärschlüssel
  \item Kommunikation über die \emph{Structured Query Language} (SQL)
  \item Datenmodellierung legt die sachlogischen Zusammenhänge über \emph{Relationen} der Tabellen fest
\end{itemize}

\begin{description}
  \item [{Datenmodell}]
	Formalismus, um interessierende Daten und sachlogische Zusammenhänge zu beschreiben.
  \item [{Datenbankschema}]
	definiert die Metastruktur, die alle möglichen Ausprägungen des konkreten Modells beschreibt.
	Das DBMS überwacht die Übereinstimmung zwischen dem Schema und dem Zustand, d.\,h., das DBMS benötigt mehrere Datenbestände.
\end{description}

\subsection{Drei-Ebenen-Konzept}
\begin{description}
  \item [{Logische~Datenunabhängigkeit}]
	Die logische Struktur der Datenbank ist unabhängig von der Schnittstelle:
	Das Datenbanksystem kann verändert werden, ohne dass die Anwendungen angepasst werden müssen.
  \item [{Physische~Datenunabhängigkeit}]
	Das Datenmodell ist unabhängig vom verwendeten Speicher(-gerät).
\end{description}

\subsubsection{Konzeptuelle Ebene}
\begin{itemize}
  \item Unternehmensweite Aufgabe, Möglichkeit zur Kontrolle stabiler Bezugspunkte
  \item Beschreibt die Gesamtheit der Daten
  \item Wird durch datenbankspezifisches Schema realisiert
      \begin{itemize}
	\item Ergebnis des Datenbankentwurfs
	\item Formuliert im DBMS des Betriebssystems
	\item Keine Details der Speicherung
	\item Keine anwendungsspezifischen Details
	\item Spezifiziert mit Hilfe der \emph{Data Definition Language} (DDL)
	\item ??? Backspace{} (nicht entzifferbar) mit separaten Anweisungen\Todo{Fix me}
	\item Zugriffsregeln spezifizieren alle relevanten Daten und Beziehungen
      \end{itemize}
\end{itemize}

\subsubsection{Interne Ebene}
\begin{itemize}
  \item Daten werden in den Speicher der Datenbank abgelegt
  \item Subformat\slash{}Zugriffsmöglichkeiten werden geklärt
  \item Beschreibt physische Speicherung (Aufbau der Datensätze, Zugriffsmethode)
  \item Bestimmt das Leistungsverhalten des Datenbanksystems
  \item Zugriffsmechanismen, Informationen
\end{itemize}

\subsubsection{Externe Ebene}
\begin{itemize}
  \item Liefert dem Nutzer die relevanten Daten
  \item Übersichtlichkeit
  \item Datenschutz
  \item Beschreibt die Sichten einzelner Anwendungsgruppen
\end{itemize}

\subsection{Aufgaben eines DBMS}
\begin{itemize}
  \item Datenbankmanagement und Zugriffsstrukturen mittels formalisierter Sprache
  \item Ausliefern und Speichern von Daten mittels Anfragen
  \item Redundanzfreie Datenhaltung des gesamten Datenbestandes
  \item Sammeln statischer Daten (Optimierung)
  \item Zentrale Konsistenzsicherung
	\begin{itemize}
	  \item Nur konsistente Änderungen möglich
	  \item Elementares Prinzip der unkollierten Änderung
	\end{itemize}
  \item Unterstützung Mehrbenutzerbetrieb (Sicherstellen der Nebenläufigkeit, Skalierbarkeit)
  \item Datenschutz und Sicherheit
  \item Einfache und flexible Nutzung der Daten
  \item Effizienz
\end{itemize}

\section{Transaktionen}

Abgeschlossene Folge von Befehlen, welche die Datenbank von einem
logisch konsistenten Zustand in einen anderen überführt. Ein Arbeitspaket,
so klein wie möglich, aber so groß wie nötig (\emph{Integritätsbedingung}).

Kernbedingungen:
\nomenclature{ACID}{Atomarity, Consistency, Isolation, Durability---Kernprinzipien eines DBMS}
\begin{labeling}{AA}
  \item [{A}] \foreignlanguage{english}{Atomarity} (Atomarität)---Entweder ist alles erfolgreich,
	      oder die Änderungen werden zurück gerollt (\foreignlanguage{english}{\emph{rollback}}).
  \item [{C}] \foreignlanguage{english}{Consistency} (Konsistenz)---Die Integritätsregeln müssen beim Abschluss der Transaktion erfüllt sein.
  \item [{I}] \foreignlanguage{english}{Isolation} (Nebenläufigkeit)---Mehrere gleichzeitige Transaktionen müssen zum gleichen
	      Ergebnis führen (Serialisierbarkeit).
  \item [{D}] \foreignlanguage{english}{Durability} (Dauerhaftigkeit)
\end{labeling}

\subsection{Zwei-Phasen-Sperrprotokoll}
\begin{itemize}
  \item Objekte vor Nutzung sperren
  \item Transaktion fordert Sperre, die sie bereits hat, nicht erneut an
  \item Transaktion respektiert vorhandene Sperren gemäß Verträglichkeitsmatrix und wird in eine Warteschlange eingereiht
  \item Spätestens bei Abschluss werden alle Sperren freigegeben.
\end{itemize}
\Todo{Grafik Sperrprotokoll}


\subsection{Dauerhaftigkeit}
\begin{itemize}
	\item Recoverysystem garantiert Atomarität und Dauerhaftigkeit.\footnote{%
		Das Recoverysystem nutzt Journaling in Form von Log-Dateien, siehe
		Seite \pageref{sub:Journaling}.%
	}
	\item Überprüfung der Konsistenzregeln am Ende entscheidet über gesamten
		Erfolg/Misserfolg
\end{itemize}

\subsection{Transaktionsfehler}
\begin{itemize}
\item Rollbackbefehl durch Anwendung
\item Verletzung von Integritäts-\slash{}Zugriffsrechten
\item Abbruch zur Auflösung von Blockaden (\foreignlanguage{english}{\emph{Deadlock}})
\end{itemize}

\subsection{Systemfehler}
\begin{itemize}
\item Stromausfall
\item Betriebssystemabsturz
\item Sonstige Hardware- oder Speicherfehler
\end{itemize}

\subsection{Plattenfehler}
\begin{itemize}
\item Physische Fehler der Platten
\item Hardwarefehler (bspw.~Controller defekt)
\end{itemize}

\subsection{Protokollierung}
\begin{description}
\item [{Logische~Protokollierung}] Aufzeichnung von Undo-\slash{}Redo-Operationen
\item [{Physische~Protokollierung}] Snapshots des Systems zu bestimmten
Zeitpunkten
\end{description}

\subsection{Behebung von Transaktionsfehlern}

\ldots{}durch Neustart des Systems und Rollback der letzten aktiven
Transaktion sowie Einspielung der gesicherten Protokolle.


\chapter{Grundlagen Datenbankentwurf}
\begin{itemize}
\item Orientiert an ANSI-SPARC-Modell
\item Analytisch
\item IT-Kenntnisse sind erst beim konzeptuellen Modell benötigt\end{itemize}
\begin{enumerate}
\item Anforderungsanalyse zur informellen Beschreibung des Fachproblems

\begin{itemize}
\item Struktur, Abhängigkeiten, Operationen, Anwendung
\end{itemize}
\item Konzeptueller Datenbankentwurf zur formalen High-Level-Beschreibung
der Daten und deren Beziehungen

\begin{itemize}
\item Formalisieren der Entwürfe
\item Semantische Struktur festlegen
\end{itemize}
\item Logischer Datenbankentwurf zur Erstellung der grundsätzlichen Metadatenstruktur
des DBMS

\begin{itemize}
\item Übersetzung des konzeptuellen Entwurfs in die Modellierungskonzepte
des konkreten DBMS
\item Formalisierung des Verfahrens zur Vermeidung schlechter Entwürfe (\emph{Anomalien})
\end{itemize}
\item Entwurf externer Sichten für Transformationsregeln

\begin{itemize}
\item Anwendungsspezifische Fortsetzung
\end{itemize}
\item Physischer Entwurf zur Planung und Errichtung von Zugriffspfaden

\begin{itemize}
\item Optimierte Abbildung des logischen Datenmodells auf interne Strukturen
und Organisationsformen des DBMS zur Durchsatzoptimierung
\end{itemize}
\end{enumerate}

\chapter{Datenbankmodelle}

\nomenclature{DBM}{Datenbankmodell; beschreibt, wie konkrete Datensätze einer Datenbank miteinander in Verbindung stehen}Ein
Datenbankmodell (DBM) beschreibt, wie konkrete Datensätze einer Datenbank
miteinander in Verbindung stehen.


\section{Komponenten}

Nach \noun{Codd} besteht ein Datenbankmodell aus drei Komponenten:
\begin{enumerate}
\item Generische Metadaten zur Beschreibung der Struktur der Daten.
\item Eine Menge von Operatoren zur Verarbeitung des Datenbestands.
\item Integritätsbedingungen.
\end{enumerate}

\section{Wichtige Modelle}
\begin{itemize}
\item Hierarchisches DBM
\item Netzwerk-DBM
\item Relationales DBM
\item Objektorientiertes DBM
\item Objektrationales DBM
\end{itemize}

\section{Das \foreignlanguage{english}{Entity-Relationship}-Modell}
\begin{itemize}
\item Formales Verfahren zur Erstellung eines konzeptuellen Modells
\item Entwickelt 1976 von \noun{Chen}, seitdem mehrfach verbessert
\selectlanguage{english}%
\item Entity-Relationship\foreignlanguage{ngerman}{-Diagramme}
\end{itemize}

\subsection{Elemente}

\begin{center}
\begin{tabular}{rcl}
Was interessiert? & \lyxarrow{} & Entities und Entity-Sets \\
Ihre Eigenschaften? & \lyxarrow{} & Attribute und Domänen (Wertebereiche) \\
Identifikation? & \lyxarrow{} & Schlüssel und Primärschlüssel \\
Sachlogische Zusammenhänge? & \lyxarrow{} & Beziehungen und deren Kardinalität
\end{tabular}
\end{center}


\subsection{Definitionen}
\begin{description}
\item [{Entity}] Konkretes Objekt, für welches Sachverhalte festzustellen
sind.\footnote{
	Bspw.~Abteilungen, Vorlesungen, Kostenstellen,\ldots{}
}
\item [{Entity-Set}] Gleichartige Entitäten (z.\,B.~Menge von Studenten).
\item [{Entitätstypen}] Menge von Entitäten mit gleicher Struktur der Merkmale.
\item [{Attribute}] Charakterisierung der Entität, atomare Werte; jede
Entität muss über eine Attributmenge eindeutig identifizierbar sein.
\item [{Domäne}] Typisierte Menge von Werten (bspw.~alle Zahlen von 1
bis 42).
\item [{Beziehungen}] Sachlogische Zusammenhänge zwischen zwei oder mehr
Entitäten\slash{}Attributen, auch rekursiv möglich.
\item [{Kardinalität}] Multiplizität der Beziehung (vgl.~UML: Aggregation
und Komposition).
\item [{Superschlüssel}] Beliebige Menge von Attributen, die eine Entität
eindeutig identifizieren.
\item [{Schlüsselkandidat}] Beliebiger Superschlüssel mit minimaler Anzahl
an Attributen, der zur eindeutigen Identifizierung benutzt werden
kann.
\item [{Primärschlüssel}] Minimale Menge von Attributen, die eine Entität
eindeutig identifizieren; beliebiger Schlüsselkandidat.
\end{description}

\subsection{\protect\noun{Chen}-Notation}
\begin{description}
\item [{1:1-Beziehung}] Jede Entität steht mit genau einer anderen Entität
in Beziehung (,,A ist verheiratet mit B``).
\item [{1:n/n:1-Beziehung}] Jede Entität hat genau eine ,,Elternentität``,
bzw.~jede Entität hat beliebig viele (auch 0) ,,Kinder``.
\item [{m:n-Beziehung}] Jede Entität steht mit beliebig vielen anderen
Entitäten in Beziehung (z.\,B.~,,Student hört Vorlesungen`` und
,,Vorlesung besteht aus Studenten``).
\end{description}
Zwischen den Entitätstypen können auch mehrere Beziehungen unterschiedlichen
Typs existieren.

\Todo{Grafiken Chen-Notation}


\subsection{(1,c,m)-Notation}
\begin{description}
\item [{1}] Einfache 1:1-Beziehung.
\item [{c}] Choice; Beziehung ist möglich, aber nicht nötig (UML: $0..1$-Multiplizität).
\item [{m}] Multiple; Beziehung mit beliebiger Kardinalität $\geq1$ ist
möglich (UML: $1..*$).
\item [{mc}] Multiple\slash{}Choice; Beziehung mit beliebiger Kardinalität
ist möglich (UML: $*$).
\end{description}

\Todo{Grafik (1,c,m)-Notation}


\subsection{Min-Max-Notation}

Notation der Kardinalität als Tupel, z.\,B.~$(1;3)$.

\emph{Achtung}: Die Kardinalitäten werden jeweils auf der \emph{anderen
Seite} der Beziehung notiert!

\Todo{Grafik Min-Max-Notation}


\subsection{Ternäre und n-stellige Beziehungen}

Ausdruckskraft: Die Kardinalitätsangaben der klassischen ER-Diagramme
und (min,max)-Notation ist bei $n$-stelligen Beziehung mit $n>2$
nicht vergleichbar.

\Todo{Grafik ternäre und n-stellige Beziehungen}


\subsection{Is-A-Beziehung}
\begin{description}
\item [{Total~(t)}] Es gibt außer den genannten Spezialisierungen keine
weiteren.
\item [{Partiell~(p)}] Neben den genannten Spezialisierungen kann es weitere
geben.
\item [{Disjunkt}] Die Spezialisierungen schließen sich gegenseitig aus
(Pfeil zur Spezialisierung); z.\,B.~ist eine Person entweder ein
Mann oder eine Frau.
\item [{Nicht~disjunkt}] Die Spezialisierungen schließen sich \emph{nicht}
gegenseitig aus (Pfeil zum Obertyp); z.\,B.~kann eine Person gleichzeitig
ein Mann und ein Angestellter sein.
\end{description}
\Todo{Grafik is-a-Beziehung}


\subsection{Assoziationen}

UML, einfache Linie, evtl.~mit Assoziationsklasse.

\Todo{Grafik UML-Assoziation}


\subsection{Aggregationen}

UML, Aggregation durch leere Raute, Komposition durch gefüllte Raute.

\Todo{Grafik UML-Aggregation}


\section{Relationales Datenbankmodell}
\begin{description}
\item [{Relation}] Seien $D_{1},\ldots,D_{n}$ Mengen von Werten, so ist
$R\subseteq D_{1}\times\cdots\times D_{n}$ eine $n$-stellige Relation
über diese Mengen.
\item [{Relationsschema}] beschreibt die Struktur des DBMS durch die Angabe
aller möglichen Relationen ($D_{1}\times\cdots\times D_{n}$, kartesisches
Produkt).
\item [{Geordnetes~Relationsschema}] Attribute können anhand ihrer Position
im Tupel referenziert werden, die Position \emph{kann} auch (eindeutig)
benannt sein (z.\,B.~,,Spalte \quotesinglbase id``` statt ,,Spalte
0``, $R=(A_{1}:D_{1},\ldots,A_{k}:D_{k})$).

\begin{description}
\item [{Vorteile}] Notation ist prägnant, Änderungen der Daten sind leichter\slash{}freier
durchzuführen
\item [{Nachteile}] Einschränkung bei logischer Datenverarbeitung, neue
Attribute dürfen die bisherige Ordnung nicht tangieren
\end{description}
\item [{Ungeordnetes~Relationsschema}] Entspricht dem geordneten Relationsschema,
jedoch ohne die Möglichkeit, Attribute über ihren numerischen Index
referenzieren zu können.

\begin{itemize}
\item $R$ ist eine Menge von Attributnamen $A_i \to D_i$ ($A_i$
ist einer Domäne zugeordnet).
\item Die Namen $A_{i}$ sind Referenzen und eindeutig.
\end{itemize}
\item [{Schlüsselkandidaten}] müssen eindeutig und minimal sein.
\item [{Geschachtelte~Relationen}] Attribute in Relationen können selbst
wieder Relationen sein.
\item [{Referenz-Attribute}] können Referenzen als Wert haben (alternative
Nutzung von Fremdschlüsseln).
\end{description}

\subsection{Anomalien und Normalformen}


\subsubsection{Anomalien}


\subsubsection{Funktionale Abhängigkeiten}


\subsubsection{\protect\noun{Armstrong}-Axiome}


\subsubsection{Bestimmung der Hülle}


\subsubsection{Kanonische Überdeckung}


\subsubsection{Erste Normalform}

Alle Attribute sind atomar (in relationalen Datenbanken nur bei Verwendung von bspw.~\sqli!GROUP BY! indirekt nicht erfüllt, in relationalen Datenbanken sind die Attribute immer atomar).


\subsubsection{Zweite Normalform}

Alle Attribute sind funktional nur vom gesamten Schlüssel abhängig,
nicht nur von einem Teil.
\begin{itemize}
\item Erfordert erste Normalform.
\item Jedes Attribut muss voll funktional abhängig von allen Schlüsselkandidaten
oder Teil eines Schlüsselkandidaten sein.
\item Nur verletzbar bei mehr als einem Attribut als Schlüsselkandidat.
\end{itemize}

\paragraph*{Transformation}
\begin{itemize}
\item Wenn die Attribute voll funktional abhängig vom Schlüsselkandidaten
sind: Ursprungsrelation $R$ beibehalten.
\item Sonst $A_{i}\to B_{i}$:

\begin{enumerate}
\item Lösche $B_{i}$ aus $R$.
\item Nach den linken Seiten gruppieren.
\item Für jede Gruppe eine neue Relation $R'$ mit $A_{i}\to B_{i}$ bilden.
\item $A_{i}$ als Schlüssel von $R'$ nehmen.
\end{enumerate}
\end{itemize}
Dies ergibt möglichst kompakte Schlüssel und beseitigt einige Anomalien,
aber entitätsfremde Daten werden in der Relation u.\,U.~nicht beseitigt.


\subsubsection{Dritte Normalform}

Keine nicht-transitive funktionale Abhängigkeit eines Nicht-Schlüssel-Attributs
von Nicht-Schlüssel-Attributen (es sollte jedes Nicht-Schlüssel-Attribut
nur ein einziges Mal im gesamten Schema in seiner entsprechenden Bedeutung
vorliegen).


\paragraph*{Synthesealgorithmus}
\begin{enumerate}
\item Kanonische Überdeckung $F_C$ bestimmen.
\item Für jede funktionale Abhängigkeit $\alpha\mapsto\beta\in F_{C}$:

\begin{enumerate}
\item $R_{\alpha}:=\alpha\cup\beta$
\item $F_{\alpha}:=\{ \alpha'\mapsto\beta'\in F_{C}\mid\alpha'\cup\beta'\subseteq R_{\alpha}\} $
\end{enumerate}
\item Abgeschlossen falls ein $R_{\alpha}$ aus 2.~einen Schlüsselkandidaten
aus $R$ bzgl.~$F_{C}$ enthält, sonst $R_{R}:=R$, $F_{R}:=\emptyset$.
\item Eliminiere alle $R_{\alpha}$, die in $R_{\alpha'}$ enthalten sind,
also $R_{\alpha}\subseteq R_{\alpha'}$.
\end{enumerate}
Nicht vermeidbar sind Abhängigkeiten mit Schlüsselkandidaten.


\subsubsection{\protect\noun{Boyce}-\protect\noun{Codd}-Normalform}

Dritte Normalform erweitert um die Bedingung, dass auch die Schlüsselkandidaten-Attribute
funktional unabhängig sind.
\begin{itemize}
\item Bei allen nicht-trivialen $\alpha\mapsto\beta$ mit $\beta\notin\alpha$
enthält $\alpha$ einen Schlüsselkandidaten von $R$.
\item Anders als bei der dritten Normalform kann ein Teil eines Schlüsselkandidaten
\emph{nicht} funktional abhängig von einem anderen Teil des Schlüsselkandidaten
sein.
\end{itemize}

\chapter{SQL}

\nomenclature{SQL}{Structured Query Language; Standardisierte Sprache zur Kommunikation mit relationalen DBMS}
\begin{itemize}
\item Standardisierte Sprache für relationale Datenbanken, basierend auf
Schlüsselworten
\item 1970 erste Ansätze, seit 1986 ANSI-Standard und immer wieder weiter
entwickelt
\item Aktuelle DBS unterstützen weitestgehend SQL-3 und SQL-03
\item Keine Unterscheidung zwischen Groß- und Kleinschreibung
\item Mengenorientiert: Arbeitet auf Tabellen, nicht auf einzelnen Daten
\item Basiert auf relationaler Algebra und ist relational vollständig
\item SQL beschreibt, \emph{was} zu machen ist, das DBMS beschreibt, \emph{wie}
es zu machen ist
\item Es gibt günstige und ungünstige Abfragen\end{itemize}
\begin{description}
\item [{Direct~SQL}] Eingabe in Shell-ähnliche Oberflächen des DBMS, ergänzt
durch Einsatz von Web-Portalen (bspw.~phpMyAdmin).
\item [{Embedded~SQL}] Einbettung von SQL-Statements in die Wirtssprache.
\item [{Call~Level~Interface~(CLI)}] Programmierschnittstelle für Datenbankzugriffe,
i.\,d.\,R.~SQL (oder ein Dialekt davon), z.\,B.: JDBC, Hibernate.
\end{description}

\section{Datentypen}

\begin{sqlcode}
INTEGER; INT; -- 32 Bit
SMALLINT; -- 16 Bit
TINYINT; MEDIUMINT; BIGINT; -- Nur MySQL
DECIMAL; DEC; NUMERIC; -- Dezimalzahlen
DEC(6,2); -- 6 Ziffern, 2 Nachkommastellen
FLOAT; DOUBLE; REAL;
CHARACTER; CHAR; -- Einzelnes Zeichen
CHAR(4); -- Zeichenkette mit genau 4 Zeichen
VARCHAR(4);-- Zeichenkette mit maximal 4 Zeichen
TEXT; -- 65535 Zeichen max.
MEDIUMTEXT; -- 16.7 Mio. Zeichen max.
LONGTEXT; -- 4 Mrd. Zeichen max.
BLOB; -- Binary Large Object, für Binärdaten
DATE; TIME; DATETIME; YEAR; TIMESTAMP; -- Für Datum/Zeit
\end{sqlcode}



\subsection{NULL}
\begin{itemize}
  \item Bei Vergleichen \emph{immer} \code{false}.
  \item Ausdrücke, die \sqli!NULL! enthalten, geben \emph{immer} \sqli!NULL! zurück.
  \item Um nach \sqli!NULL! zu suchen, ist \sqli!IS NULL! zu verwenden.
  \item Bei \sqli!ORDER BY! werden \sqli!NULL!-Werte zuerst aufgeführt.
  \item Bei \sqli!ORDER BY! werden \sqli!NULL!-Werte gleich behandelt.
\end{itemize}

\subsection{Wahrheitstabellen}

w: Wahr, f: Falsch; n: \sqli!NULL!

\begin{center}
\begin{tabular}{cccc}
\bfseries \code{AND} & \bfseries w & \bfseries f & \bfseries n \\
\bfseries w & w & f & n \\
\bfseries f & f & f & f \\
\bfseries n & n & f & n
\end{tabular}
\hfill
\begin{tabular}{cccc}
\bfseries \code{OR} & \bfseries w & \bfseries f & \bfseries n \\
\bfseries w & w & w & w \\
\bfseries f & w & f & n \\
\bfseries n & w & n & n
\end{tabular}
\hfill
\begin{tabular}{cc}
\bfseries \code{NOT} & \\
\bfseries w & f \\
\bfseries f & w \\
\bfseries n & n
\end{tabular}
\end{center}

\selectlanguage{english}%

\section{Data Retrieval Language\foreignlanguage{ngerman}{ (DRL)}}

\selectlanguage{ngerman}%
\nomenclature{DRL}{Data Retrieval Language; Untersprache der SQL zur Abfrage von Daten}\ldots{}ist
die häufigste Nutzungsart, dient zur Abfrage von Daten.

SELECT, WHERE, GROUP BY, HAVING, CASE


\subsection{Algebra}

\textbf{\emph{Wichtig!}} Bei Verbund, Vereinigung, Differenz und Durchschnitt müssen die Schemata der Relationen gleich oder zumindest kompatibel sein.
\begin{description}
  \item [{Vereinigung}] $R=S\cup T$
  \begin{itemize}
    \item Ohne Eliminierung: \sqli!SELECT * FROM s UNION ALL SELECT * FROM t!
    \item Beschränkung auf gleichnamige Attribute: \sqli!SELECT * FROM s UNION CORRESPONDING SELECT * FROM t!
  \end{itemize}
  \item [{Differenz}] $R=S-T$ oder $R=S\setminus T$
  \begin{itemize}
    \item Ohne Eliminierung: \sqli!SELECT * FROM s EXCEPT ALL SELECT * FROM t!
    \item Mit Eliminierung: \sqli!SELECT * FROM s EXCEPT SELECT * FROM t!
    \item Statt \sqli!EXCEPT! wird in einigen Datenbanksystemen auch \sqli!MINUS! verwendet.
  \end{itemize}
  \item [{Kartesisches~Produkt}] $R=S\times T$
  \item [{Selektion}] $R=\sigma_{F}(S)$ (\sqli!SELECT DISTINCT * FROM s WHERE f!)
  \item [{Projektion}] $R=\pi_{A,B,\ldots}(S)$ (\sqli!SELECT DISTINCT a,b FROM s!)
  \item [{Kommutativ}] Falls $P$ nur $R1$-Attribute enthält: $\pi_{R1}\sigma_{P}(R)=\sigma_{P}\pi_{R1}(R)$, $R\bowtie S=S\bowtie R$
  \item [{Assoziativ}] $R\bowtie(S\bowtie T)=(R\bowtie S)\bowtie T$
  \item [{Distributiv}] $R\bowtie(S\cup T)=(R\bowtie S)\cup(R\bowtie T)$
  \item [{Idempotent}] Falls $R1\subseteq R2$: $\pi_{R1}(\pi_{R2}(R))=\pi_{R1}(R)$, $\sigma_{P1}(\sigma_{P2}(R))=\sigma_{P1\cap P2}(R)$
\end{description}

\subsubsection{Ableitbare Operationen}
\begin{description}
  \item [{Verbund}] $R\bowtie_{G}S:=\sigma_{G}(R\times S)$ (\sqli!SELECT * FROM r FULL OUTER JOIN s WHERE g!)
  \item [{Halbverbund}] $R\ltimes_{G}S$ (\sqli!SELECT * FROM r LEFT OUTER JOIN s WHERE g!)
  \item [{Durchschnitt}] $R=S\cap T$
  \begin{itemize}
    \item Mit Eliminierung: \sqli!SELECT * FROM s INTERSECT SELECT * FROM t!
    \item Ohne Eliminierung: \sqli!SELECT * FROM s INTERSECT ALL SELECT * FROM t!
  \end{itemize}
  \item [{Division}] $R=S\div T$, $(R\div S)\times T\subseteq S$
\end{description}

\subsection{Quantoren}
\begin{description}
  \item [{Existenzquantor}] $\{x\in R\mid\exists y\in S:\ldots\}\equiv\pi_{R}\bigl(\sigma_{\ldots}(R\times S)\bigr)$
  \item [{Allquantor}] $\{x\in R\mid\forall y\in S:\ldots\}\equiv\bigl(\sigma_{\ldots,(R)}\bigr)\times S$
\end{description}

\subsection{Reihenfolge der Ausführung}

Bei einem Statement der Form \sqli!SELECT * FROM s WHERE b! werden die Operationen in folgender Reihenfolge ausgeführt:
\begin{itemize}
  \item Die Relation wird mit \sqli!FROM s! ausgewählt.
  \item Das optionale \sqli!WHERE b! wird zur Filterung ausgeführt.
  \item Die Attribute werden mit \sqli!SELECT *! ausgewählt.
\end{itemize}

\subsection{Vereinigung}


\subsubsection{Outer Join}
\begin{description}
  \item [{Left~Outer~Join}] Es werden die nicht verbindbaren Tupel der linken Seite aufgenommen, d.\,h., die linke Tabelle wird komplett ins Ergebnis übertragen und die rechte Seite des Verbunds ggf.~mit \sqli!NULL! aufgefüllt.
  Notation: $R=S\ltimes T$, \sqli!SELECT * FROM a LEFT OUTER JOIN b ON(a.id=b.id)! bzw.~\sqli!SELECT * FROM a LEFT OUTER JOIN b USING(id)!.
  \item [{Right~Outer~Join}] Analog: $R=S\rtimes T$, in SQL \sqli!RIGHT OUTER JOIN!.
   \item [{Full~Outer~Join}] Vereinigung von Left Outer Join und Right Outer Join: $R=S\bowtie T$, in SQL \sqli!FULL OUTER JOIN!.
\end{description}

\subsubsection{Inner Join}

Es werden keine nicht-verbindbaren Einträge ins Ergebnis übernommen, in SQL \sqli!INNER JOIN!.


\subsubsection{Natural Join}

Wie Inner Join, aber es werden Verbundspalten mit gleichen Namen reduziert (d.\,h., sie kommen nicht doppelt vor), in SQL \sqli!NATURAL JOIN!.


\subsection{Funktionen}

NVL, AVG, SUM, MIN, MAX, COUNT

\begin{sqlcode}
SELECT COUNT(*) FROM tbl;
SELECT SUM(salary) FROM tbl;
SELECT AVG(salary), department_id
  FROM tbl
  GROUP BY department_id;
SELECT AVG(salary), department_id
  FROM tbl
  WHERE name='Schulze'
  GROUP BY department_id
  HAVING SUM(salary)>100;
SELECT NVL(name,'unbekannt') FROM customers;
SELECT * FROM tbl WHERE id BETWEEN 1 AND 10;
SELECT * FROM tbl WHERE id IN (2,5,7);
\end{sqlcode}



\subsection{Unterabfragen}

IN, ANY, ALL, EXISTS
\begin{description}
  \item [{Korrelliert}] Die Unterabfrage verwendet Attribute der äußeren Abfrage; kann zu einer verschachtelten Abfrageschleife führen, daher meist ungünstig.
\end{description}


\sqli!SELECT * FROM a WHERE a.id IN (SELECT id FROM b WHERE c)!

\section{Data Manipulation Language (DML)}

\nomenclature{DML}{Data Modification Language; Untersprache der SQL zur Veränderung von Daten}
\ldots dient zur Manipulation und Abfrage von Belegungen.

DELETE, INSERT, UPDATE

\begin{sqlcode}
INSERT INTO tbl VALUES(...); -- geordnet
INSERT INTO tbl(col1,col2) VALUES(123,456); -- ungeordnet
DELETE FROM tbl WHERE id=3;
UPDATE tbl SET col=val;
UPDATE tbl SET col=val WHERE id=456;
\end{sqlcode}


\section{Data Definition Language (DDL)}

\nomenclature{DDL}{Data Definition Language; Untersprache der SQL zur Definition von Datenstrukturen}
\ldots dient zur Daten- und Sichtendefinition. 

\begin{sqlcode}
CREATE TABLE tbl (
  id INT PRIMARY KEY,
  name VARCHAR(40),
  subkey INT UNIQUE NOT NULL DEFAULT 1,
  chk INT CHECK(fk BETWEEN 1 AND 100),
  CONSTRAINT fk_spec FOREIGN KEY(subkey) REFERENCES other_tbl(other_col)
);
CREATE TABLE IF NOT EXISTS tbl (
  id INT PRIMARY KEY,
  ref INT REFERENCES other_tbl(other_col) ON UPDATE CASCADE
);
ALTER TABLE tbl ADD(txt CHAR(4));
ALTER TABLE tbl MODIFY(txt CHAR(10));
ALTER TABLE tbl DROP COLUMN txt;
ALTER TABLE tbl RENAME id TO xxx;
DROP TABLE tbl;

CREATE OR REPLACE VIEW vw AS
  SELECT txt FROM tbl;

SELECT (
  CASE WHEN note<2
  THEN 'Gut'
  ELSE 'Versagt'
  END
) FROM noten;
\end{sqlcode}


CREATE TABLE, ALTER TABLE, DROP TABLE, RENAME, CONSTRAINT, TRIGGER, CREATE VIEW

\section{Data Control Language (DCL)}

\nomenclature{DCL}{Data Control Language; Untersprache der SQL zur Definition von Integritäts-, Zugriffs- und Transaktionskontrolle}
\ldots dient zur Definition von Integritäts-, Zugriffs- und Transaktionsbedingungen.

PRIMARY KEY, FOREIGN KEY, CHECK


\chapter{Das interne Modell}


\chapter{Anwendungsinteraktion}


\section{JDBC}
